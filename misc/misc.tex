本人将一些基本的矩阵类型以及相关算法写为\textsf{C++}头文件以方便调用。
本附录将简要介绍可用的接口。
所有的类以及函数均定义于命名空间\texttt{Misc}中,故下文不将\texttt{Misc::}显式写出。

可以通过引入\texttt{"Matrix\_Catalogue.h"}头文件引入所有矩阵类型以及有关运算符重载。
可以通过引入\texttt{"linear\_eq\_direct.h"}与\texttt{"linear\_eq\_iterative.h"}头文件调用矩阵的直接解法与迭代解法。

\section{矩阵}
所有的矩阵类型都继承自虚基类\texttt{template <typename T, size\_t R, size\_t C> class Base\_Matrix},其中模板参数\texttt{T}为数据类型,\texttt{R}, \texttt{C}分别为行数和列数。
注意,\textbf{所有矩阵规模参数都需要编译时指定}。
继承树为
{
\small
\begin{verbatim}
Base_Matrix<T, R, C> 虚基类
|Band_Matrix<T, N, M> 长宽为N,半带宽为M的带状矩阵
||Base_Half_Band_Matrix<T, N, M> 虚基类,长宽为N,半带宽为M
|||Low_Band_Matrix<T, N, M> 长宽为N,半带宽为M的下三角带状矩阵
|||Symm_Band_Matrix<T, N, M> 长宽为N,半带宽为M的对称带状矩阵
|||Up_Band_Matrix<T, N, M> 长宽为N,半带宽为M的上三角带状矩阵
|Base_Tri_Matrix<T, N> 虚基类,长宽为N
||Hermite_Matrix<T, N> 厄密矩阵,长宽为N
||Low_Tri_Matrix<T, N> 下三角矩阵,长宽为N
||Symm_Matrix<T, N> 对称矩阵,长宽为N
||Up_Tri_Matrix<T, N> 上三角矩阵,长宽为N
|Diag_Matrix<T, N> 对角矩阵,长宽为N
|Matrix<T, R, C> 一般矩阵,R行C列
|Sparse_Matrix<T, R, C> 稀疏矩阵;同时继承自std::array<std::map<size_t, T>, R>
\end{verbatim}
}

所有矩阵均提供的接口有
{
\linespread{1.0}
\begin{lstlisting}
Row<T, R, C> row(size_type pos);
const Row<T, R, C> row(size_type pos) const; // 返回某一行
Row<T, R, C> operator[](size_type pos);
const Row<T, R, C> operator[](size_type pos) const; // 同row()
Column<T, R, C> column(size_type pos);
const Column<T, R, C> column(size_type pos) const; // 返回某一列

const Transpose<T, C, R> trans() const; // 返回仅作为右值的转置

T &operator()(size_type row, size_type col);
const T &operator()(size_type row, size_type col); // 返回元素

using p_ta = T (*)[C];
p_ta data();
const p_ta data() const; // 返回内部存储数据的指针
constexpr size_type size() const; // 返回作为一个矩阵的元素个数
size_type data_size() const; // 返回存储的元素个数
size_type data_lines() const; // 返回存储的元素行数
constexpr size_type rows() const; // 返回矩阵行数
constexpr size_type cols() const; // 返回矩阵列数
std::pair<size_type, size_type> shape() const; // 返回矩阵形状
\end{lstlisting}
}

\texttt{row()}和\texttt{column()}方法返回\texttt{Row<T, R, C>}或\texttt{Column<T, R, C>}对象,可作为左值或右值。
可以通过\texttt{[]}下标运算符取或修改原矩阵的元素。

使用标准库类型\texttt{std::array<T, N>}作为矢量。

重载了矩阵间、矩阵与\texttt{Row}或\texttt{Column}或\texttt{array}之间的\texttt{*}运算符,可进行矩阵乘法。

所有矩阵均支持默认初始化,可以提供一个默认值。
如不提供,默认为0。

定义了一些类型间转换,可以把一些特殊的矩阵转化为更一般的矩阵。
比如,\texttt{Diag\_Matrix}可以转换为\texttt{Band\_Matrix}、\texttt{Up\_Tri\_Matrix}等等矩阵。

同时还定义了通过初始化器的列表初始化以便字面指定矩阵。
下面主要介绍一下这一初始化的使用。

\subsection{带状矩阵\texttt{Band\_Matrix}等}
需要沿对角线方向从右上到左下逐行输入,比如
{
\small
\begin{verbatim}
{
    {4, 6, 8, 3, 7},
    {3, 4, 7, 7, 4, 4},
    {2, 5, 5, 5, 2, 1, 7},
    {6, 3, 4, 4, 3, 6},
    {2, 7, 9, 2, 5}
}
\end{verbatim}
}
会得到矩阵\[
    \mqty(
        2&3&4&0&0&0&0\\
        6&5&4&6&0&0&0\\
        2&3&5&7&8&0&0\\
        0&7&4&5&7&3&0\\
        0&0&9&4&2&4&7\\
        0&0&0&2&3&1&4\\
        0&0&0&0&5&6&7
    ).
\]
输入格式错误会引发运行时错误。

对于半带状矩阵以及对称带状矩阵,初始化器的输入顺序为\textbf{从对角线向非零一侧}。

\subsection{三角矩阵\texttt{Base\_Tri\_Matrix}等}
水平从上到下依次输入半侧元素。
比如,
{
\small
\begin{verbatim}
{
    {4},
    {3, 4},
    {2, 1, 7},
    {6, 3, 4, 6},
    {2, 7, 9, 2, 5}
}
\end{verbatim}
}
会得到下三角矩阵\[
    \mqty(
        4&&&&\\
        3&4&&&\\
        2&1&7&&\\
        6&3&4&6&\\
        2&7&9&2&5
    )
\]
或对称矩阵\[
    \mqty(
        4&3&2&6&2\\
        3&4&1&3&7\\
        2&1&7&4&9\\
        6&3&4&6&2\\
        2&7&9&2&5
    ).
\]

但对应的上三角矩阵需要这样输入,
{\small
\begin{verbatim}
{
    {2, 7, 9, 2, 5},
    {6, 3, 4, 6},
    {2, 1, 7},
    {3, 4},
    {4}
}
\end{verbatim}
}
得到\[
    \mqty(
        2&7&9&2&5\\
        0&6&3&4&6\\
        0&0&2&1&7\\
        0&0&0&3&4\\
        0&0&0&0&4
    ).
\]

\subsection{对角矩阵\texttt{Diag\_Matrix}}
直接在一个初始化器列表内输入所有对角元即可。
也可通过提供一对迭代器初始化。

\section{矩阵的直接解法}
\texttt{"linear\_eq\_direct.h"}中主要是对矩阵进行分解的算法,提供\begin{enumerate}
    \item 三角矩阵回代 \texttt{back\_sub()}
    \item LU分解 \texttt{lu\_factor()}
    \item LDL分解 \texttt{ldl\_factor()}
    \item Gauss-Jordan法求逆矩阵 \texttt{inv()};对特殊矩阵也会调用用其他算法
    \item 三对角矩阵追赶法 \texttt{tri\_factor()}
    \item Cholesky分解 \texttt{cholesky}
    \item 行列式计算 \texttt{det()}
\end{enumerate}

所有算法(除\texttt{det()})外均提供原处修改的接口与非原处修改接口。
如果仅传入待分解矩阵,那么会进行原处修改;如果也传入了输出矩阵,那么不会修改原矩阵。

\texttt{det()}函数会返回行列式的值。

\section{矩阵的迭代解法}
\texttt{"linear\_eq\_iterative.h"}中主要是对线性方程组进行迭代求解的算法,提供\begin{enumerate}
    \item Jacobi迭代法 \texttt{jacobi()}
    \item Gauss-Seidel迭代法 \texttt{gauss\_seidel()}
    \item 超松弛迭代法 \texttt{suc\_over\_rel()}
    \item 共轭梯度法 \texttt{grad\_des()}
\end{enumerate}

所有算法的接口是基本统一的,参数列表如下\begin{enumerate}
    \item \verb|const Base_Matrix<T, N, N> &in_mat| 系数矩阵
    \item \verb|const std::array<T, N> &in_b| 非齐次项
    \item \verb|std::array<T, N> &out_x| 解向量的初始值;输出的解向量
    \item \verb|T omega| 仅超松弛迭代使用的系数
    \item \verb|bool sparse = false| 是否为稀疏矩阵;如是,那么有优化
    \item \verb|size_t max_times = 1000| 最大迭代次数
    \item \verb|double rel_epsilon = 1e-15| 判停标准:残差与初始残差的1-范数之比小于此值时即停止
\end{enumerate}
