\documentclass[a4paper,unicode]{report}
\usepackage{amsmath}
\usepackage{amssymb}
\usepackage{textgreek}
\usepackage{esint}
\usepackage{subfig}
\usepackage{rotating}
\usepackage{booktabs}
\usepackage{longtable}
\usepackage{paralist}
\usepackage{titlesec}
\usepackage{graphicx}
\usepackage{physics}
\usepackage{ucs}
\usepackage{listings}
\usepackage{multirow}
\usepackage{tablefootnote}
\usepackage{indentfirst}
\usepackage{tikz}
\usepackage{warpcol}
\usepackage[pdftitle=计算物理第一次大作业,pdfauthor=林旭辰,bookmarksnumbered=true]{hyperref}
\usepackage{natbib}
\usepackage{xcolor}
\usepackage{xeCJK}
    \setCJKmainfont[BoldFont={Noto Serif CJK SC Bold},ItalicFont={FangSong}]{Noto Serif CJK SC}
    \setCJKsansfont[BoldFont={Noto Sans CJK SC Bold},ItalicFont={KaiTi}]{Noto Sans CJK SC}
    \setCJKmonofont[BoldFont={Noto Sans Mono CJK SC Bold}]{Noto Sans Mono CJK SC}

\title{计算物理第一次大作业}
\author{物理学院\quad 林旭辰\quad 1800011324}
% \date{2019年5月25日}

\DeclareUnicodeCharacter{"00B0}{\textdegree}

\renewcommand{\today}{\number\year 年\number\month 月\number\day 日}
\renewcommand{\refname}{参考文献}
\renewcommand{\abstractname}{摘要}
\renewcommand{\contentsname}{目录}
\renewcommand{\figurename}{图}
\renewcommand{\tablename}{表}
\renewcommand{\appendixname}{附录}
\renewcommand{\chaptername}{第}
\newcommand{\chapterendname}{章}

\renewcommand\appendix{\par
  \setcounter{chapter}{0}%
  \setcounter{section}{0}%
  \renewcommand\chaptertitlename{\appendixname}%
  \renewcommand\thechapter{\Alph{chapter}}%
  \renewcommand\chapterendname{}}

\renewcommand{\equationautorefname}{式}
\renewcommand{\figureautorefname}{图}
\newcommand{\subfigureautorefname}{子图}
\renewcommand{\tableautorefname}{表}
\renewcommand{\subsectionautorefname}{小节}
% \newcommand{\mythmautorefname}{定理}
\renewcommand{\sectionautorefname}{\S}

% \newtheorem{mythm}{定理}

\lstset{
    basicstyle=\scriptsize\ttfamily\color{black}, % print whole listing small
    keywordstyle=\color{teal}\bfseries,
    identifierstyle=, % nothing happens
    commentstyle=\color{gray}\ttfamily, % white comments
    stringstyle=\color{violet}, % typewriter type for strings
    showstringspaces=true,
    numbers=left,
    numberstyle=\tiny\color{brown},
    stepnumber=2,
    numbersep=5pt,
    firstnumber=auto,
    frame=lines,
    language=[11]C++,
    rangeprefix=/*,
    rangesuffix=*/,
    includerangemarker=false
}

% \pgfsetxvec{\pgfpoint{0.02cm}{0}}
% \pgfsetyvec{\pgfpoint{0}{0.1em}}

% \usetikzlibrary{datavisualization,plotmarks,datavisualization.formats.functions}
% \usetikzlibrary{math,fpu,datavisualization}
% \graphicspath{{figure/}}
\linespread{1.3}

\titleformat{\chapter}[display]{\huge\bfseries}{\chaptertitlename\ \thechapter\ \chapterendname}{20pt}{\Huge}

\begin{document}

\maketitle
\tableofcontents

\begin{center}
    \textbf{编译与运行环境说明}
\end{center}

本地操作系统版本为\textsf{Windows 10 x64 1909 (18363.720)}。

第一题电阻网络使用\textsf{C++}编写,编译器为Windows下的\texttt{g++ 9.3.0} (Rev2, Built by MSYS2 project),标准为\textsf{C++14}。
除标准库外未使用其他库。
最后用于计时等操作的程序的编译命令为\begin{verbatim}
    g++ -std=c++14 -o <可执行文件名> <源文件名>
\end{verbatim}
综合使用\textsf{GNU time}程序以及\texttt{<chrono>}库进行计时。
计时时,计算机连接电源。

第二题Logistic模型使用\textsf{Python}编写,解释器为Windows下的\texttt{Python 3.8.1} (tags/v3.8.1:1b293b6, Dec 18 2019, 23:11:46) [MSC v.1916 64 bit (AMD64)] on win32。
除为绘图而调用的\texttt{matplotlib.pyplot} (3.2.0)外未使用其他第三方库。
使用了标准库\texttt{random}以生成随机数。

\chapter{电阻网络}
本题要求使用矩阵的迭代法与直接解法对\autoref{fig:network} 所示的三类无源电阻网格的某给定两点间的电阻进行求解,并比较迭代法与直接解法的用时差别。
其中,矩阵是根据基尔霍夫第二定律
\begin{equation}\label{eq:Kirchhoff}
    \sum_{j\neq i} {g_{ij}(U_i-U_j)} = I_i,\quad i=1,2,\cdots,n
\end{equation}
得到的。
此式中各符号的含义会在后文详述。

\begin{figure}
    \centering
    \subfloat[规模为4的方型网格]{
        \label{fig:network_square}
        \includegraphics[width=.3\textwidth]{figures/square.pdf}
        }
        \subfloat[规模为4的三角型网格]{
            \label{fig:network_triangle}
        \includegraphics[width=.3\textwidth]{figures/triangle.pdf}
        }
        \subfloat[规模为4的六边形型网格]{
            \label{fig:network_hexagon}
            \includegraphics[width=.3\textwidth]{figures/hexagon.pdf}
    }
    \caption{题给的三种电阻网络}
    \label{fig:network}
\end{figure}

题目要求对多种规模(边数为1\footnote{六边形型网格对周期数为1的情况不作要求;该情况显然$R_\text{ab}=0$}、4、16、64)的各边阻值为单位1的网格,求解\autoref{fig:network} 中ab或ac\footnote{仅方型有}间的阻值。
此外,第4小问还要求做出\autoref{fig:network_triangle} 中的三角网络在水平连线均变为单位电容、斜向右下的连线变为单位电感时ab间等效阻抗的频率响应曲线。

下面,我将结合程序\verb|1_resister_network.cpp|,逐步介绍
\begin{inparaenum}
    \item 基尔霍夫第二定律;
    \item 网格抽象为矩阵的建模方法;
    \item 计算过程简述;
    \item 计算结果与耗时情况;
    \item 三角$RLC$网格的频率响应
\end{inparaenum}。

\section{基尔霍夫第二定律}

设一个无源网格中有$n$个节点,根据电荷守恒以及欧姆定律,可以合理推得由外部流入节点$i$的电流$I_i$等于由$i$流向所有与其相连的节点的总电流。
而各支路的电流(设流向节点$j$)都可利用欧姆定律表示为电势差$U_i-U_j$与导纳$g_{ij}$之积。
因而我们得到式 \eqref{eq:Kirchhoff}\begin{equation}
    \sum_{j\neq i} {g_{ij}(U_i-U_j)} = I_i,\quad i=1,2,\cdots,n.\tag{\ref{eq:Kirchhoff}}
\end{equation}

可将各节点的电势表示为一个$n$维矢量$\vb*{U}$,外界流入电流表示为$\vb*{I}$。
那么,上式可改写为\begin{equation}
    \sum_{j}\sum_{k\neq i}{g_{ik}\delta_{ij}}U_j
        -\sum_{j\neq i}g_{ij}U_j = I_i,
\end{equation}
亦即\begin{gather}
    \vb{G}\vb*{U} = \vb*{I},\\
    \vb{G}_{ij} =
    \begin{cases}
        \sum_{k\neq i}g_{ik},& i=j,\\
        -g_{ij},&   i\neq j.
    \end{cases}
    \label{eq:Gij}
\end{gather}

注意到,电势零点是可以任意选取的,这意味着$\vb{G}$的秩必为$n-1$。
通过将某点电势取为0,将该点对应的导纳矩阵行与列删去,我们得到一个非奇异矩阵。

求解某两点间电阻,即等价于在一点输入单位电流、一点输出单位电流,求两点电势差。
此即线性方程求解问题。

\section{导纳矩阵建模}
我们需要求解的电路网格见\autoref{fig:network}。
三种网格的建模思路其实都很类似,我将从最简单的方型网络开始。

\subsection{方型网格}
首先定义如何表示格点在网格中的位置。
记a点为$(0,0)$,$y$轴向上延伸,$x$坐标则以一行中最左边的格点为0。

再定义将网格变为矩阵的编号方式。
从左下角开始,取点a为电势零点,不编号;然后从左至右、从下到上依次进行编号。
\autoref{fig:network_square} 为周期数为4时的编号示例。

使用的接口声明如下:
{
    \linespread{1.0}
    \lstinputlisting[linerange=beg:square_api-end:square_api]{1_resister_network.cpp}
}
\verb|calc_square()|为具体进行计算的函数,模板参数\texttt{N}为网格空间周期数;
\verb|square_network()| 为生成导纳矩阵$\vb{G}$的函数;
\verb|_square_index()| 与 \verb|_square_xy()| 分别根据一个格点在网格中的位置得到编号或根据编号得到在网格中的位置;
\verb|_square_connections(index)|返回所有与格点\texttt{index}相连的且编号大于\texttt{index}的格点的编号的\texttt{std::vector}。

\paragraph{\texttt{square\_network()}}\label{para:sq_network}
的定义如下:
{
    \linespread{1.0}
    \lstinputlisting[linerange=beg:square_network-end:square_network]{1_resister_network.cpp}
}
如同函数前注释中解释的,$\vb{G}$的规模应为$(\text{\texttt{N}}^2+2\text{\texttt{N}})\times(\text{\texttt{N}}^2+2\text{\texttt{N}})$。
故将矩阵行数定义为常量表达式\verb|mat_length|。
同时,$\vb{G}$一定为半带宽为$\text{\texttt{N}}+1$的对称矩阵(一个格点最多与和它相差$\text{\texttt{N}}+1$的格点相连),故将返回矩阵定义为\verb|Symm_Band_Matrix<double, mat_length, N + 1>|。

对网格中编号为\verb|i|的格点进行循环。
找出与其相连的编号大于\verb|i|的格点的编号存入\verb|connections|。
那么,根据式 \eqref{eq:Gij},\verb|i|对应的矩阵对角元要增加这些格点的个数(由于阻值是单位1)。
同时,\verb|connections|中每一个\verb|j|对应一个$g_{ij}=-1$。

由于每次只考虑编号大于\verb|i|的格点,循环不重不漏,除了点$(0,0)$。
故我们再将与$(0,0)$直接相连的两点的对角元加1。

\paragraph{\texttt{\_square\_index()}与\texttt{\_square\_xy()}}
的定义如下:
{
    \linespread{1.0}
    \lstinputlisting[linerange=beg:_square_index-end:_square_index]{1_resister_network.cpp}
    \lstinputlisting[linerange=beg:_square_xy-end:_square_xy]{1_resister_network.cpp}
}

两者的实现都十分直接,不多赘述。

\paragraph{\texttt{\_square\_connections()}}
的定义如下:
{
    \linespread{1.0}
    \lstinputlisting[linerange=beg:_square_connections-end:_square_connections]{1_resister_network.cpp}
}
如果此格点不在一行的最右边,那么其与它右边的点相连;
如果此格点不在一列的最上边,那么其与它上边的点相连。

\subsection{三角型网格}\label{ssec:triangle}
首先定义如何表示格点在网格中的位置。
记a点为$(0,0)$,$y$轴向上延伸,将一行中最左边的格点的$x$坐标记为0。

再定义将网格变为矩阵的编号方式。
从左下角开始,取点a为电势零点,不编号;然后从左至右、从下到上依次进行编号。
\autoref{fig:network_triangle} 为周期数为4时的编号示例。

使用的接口声明如下:
{
    \linespread{1.0}
    \lstinputlisting[linerange=beg:triangle_api-end:triangle_api]{1_resister_network.cpp}
}
各函数的功能与方型的类似。

\paragraph{\texttt{triangle\_network()}}\label{para:tri_net}
与方型的\verb|square_network()|十分接近。
函数前的注释{
    \linespread{1.0}
    \lstinputlisting[linerange=beg:triangle_network_comment-end:triangle_network_comment]{1_resister_network.cpp}
}清楚地说明了$\vb{G}$的规模为$\frac{\text{\texttt{N}}\times(\text{\texttt{N}}+3)}{2}\times\frac{\text{\texttt{N}}\times(\text{\texttt{N}}+3)}{2}$。
类似 \nameref{para:sq_network},矩阵半带宽也为$\text{\texttt{N}}+1$。

\paragraph{\texttt{\_triangle\_index()}与\texttt{\_triangle\_xy()}}
的定义如下:{
    \linespread{1.0}
    \lstinputlisting[linerange=beg:_triangle_index-end:_triangle_index]{1_resister_network.cpp}
    \lstinputlisting[linerange=beg:_triangle_xy-end:_triangle_xy]{1_resister_network.cpp}
}\verb|_triangle_index()| 为一个简单的等差数列求和。
\verb|_triangle_xy()| 则是逐步将满的行从 \verb|index| 中减去,直到一对合理的$(x,y)$值。

\paragraph{\texttt{\_triangle\_connections()}}
的定义如下:
{
    \linespread{1.0}
    \lstinputlisting[linerange=beg:_triangle_connections-end:_triangle_connections]{1_resister_network.cpp}
}
如果此格点不在一行的最右边,那么其与它右边的点相连;
如果此格点不在一列的最上边且不在一行的最右边,那么其与它上边的点相连;
如果此格点不在一行的最左边,那么其与它左上方的点相连。

\subsection{六边形型网格}
为方便实现,六边形型网格的编号方式与前面的不大相同。
从左下角开始,仍取点a为电势零点;然后从右下至左上、从左下到右上依次进行编号。
\autoref{fig:network_hexagon} 为周期数为4时的编号示例。

下面定义如何表示格点在网格中的位置。
这段函数注释比较清楚地标明了坐标架:
{
    \linespread{1.0}
    \lstinputlisting[linerange=beg:_hex_index_comment-end:_hex_index_comment]{1_resister_network.cpp}
}
这里将\autoref{fig:network_hexagon} 中的网格顺时针旋转了120\textdegree 。
$x=0$的点为每一行中编号最小的;
点$(0,0)$为编号为0的点。

使用的接口声明如下:
{
    \linespread{1.0}
    \lstinputlisting[linerange=beg:hex_api-end:hex_api]{1_resister_network.cpp}
}
各函数的功能与方型的类似。

\paragraph{\texttt{hex\_network()}}
与方型的\verb|square_network()|十分接近。
$\vb{G}$的规模为$(\text{\texttt{N}}\times(\text{\texttt{N}}+4))\times(\text{\texttt{N}}\times(\text{\texttt{N}}+4))$。
类似 \nameref{para:sq_network},矩阵半带宽也为$\text{\texttt{N}}+1$。

点$(0,0)$与$(0,1)$和a点直接相连,结尾有考虑这一特殊情况。

\paragraph{\texttt{\_hex\_index()}与\texttt{\_hex\_xy()}}
的定义如下:{
    \linespread{1.0}
    \lstinputlisting[linerange=end:_hex_index_comment-end:_hex_index]{1_resister_network.cpp}
    \lstinputlisting[linerange=beg:_hex_xy-end:_hex_xy]{1_resister_network.cpp}
}
注意到,若将$y$为奇数与偶数的行两两凑成一对,它们的元素个数仍然组成一个等差数列$\{4,6,8,\cdots,2\text{\texttt{N}}+2\}$。
变量\verb|l|存储了$(\text{\texttt{x}},\text{\texttt{y}})$在这一“等差数列”中的位置。
由等差数列求和可知,直到上一奇数行尾止,元素个数为$\frac{(4+(2l+2))l}{2}=l^2+3l$。

然后,如果$y$为奇数,还要加上$y-1$行的$l+2$。
再加上$x$的贡献,故最终\verb|_hex_index()|得到\begin{equation}
    l(l+2)+l+y\bmod 2 \times(l+2)+x = (l+y\bmod 2)(l+2)+l+x.
\end{equation}

\verb|_hex_xy()| 则是逐步将满的行从 \verb|index| 中减去,直到一对合理的$(x,y)$值。
对于$y=2N$的情况,由于定义,$x$要减去1。

\paragraph{\texttt{\_triangle\_connections()}}
的定义如下:
{
    \linespread{1.0}
    \lstinputlisting[linerange=beg:_hex_connections-end:_hex_connections]{1_resister_network.cpp}
}

注意到,网格可以抽象为\autoref{fig:abs_hex_network}。
如果此格点编号大于27($N(N+3)-1$),那么其与$y$比它大1的点相连;
如果此格点在倒数第二行且不在最左边,那么其与$y$比它大1、$x$比它小1的点相连;
如果此格点在奇数行,那么其与$y$比它大1、$x$比它大1的点相连。

\begin{figure}
    \centering
    \includegraphics[width=0.5\textwidth]{figures/abs_hexagon.pdf}
    \caption{抽象化的六边形型网格。注意到点14与点17均可作为b}
    \label{fig:abs_hex_network}
\end{figure}

\section{电阻计算过程}
下面将介绍上面给出的各\verb|calc_*<N>()|函数的实现。
详细介绍的将是方型网格;其他网格与之十分类似。

\subsection{方型网格\texttt{calc\_square<N>()}}
\label{ssec:calc_square}
此函数为函数模板,模板参数\verb|size_t N|为空间周期数。
{
    \linespread{1.0}
    \lstinputlisting[linerange=beg:calc_square-calc_square_1]{1_resister_network.cpp}
}

下面看具体实现。
{
    \linespread{1.0}
    \lstinputlisting[linerange=calc_square_1-calc_square_2]{1_resister_network.cpp}
}
为方便,定义常量表达式\verb|mat_side|为系数矩阵边长;\verb|band_width|为半带宽。
{
    \linespread{1.0}
    \lstinputlisting[linerange=calc_square_2-calc_square_3]{1_resister_network.cpp}
}
使用前面介绍的 \nameref{para:sq_network} 生成系数矩阵$\vb{G}$。
{
    \linespread{1.0}
    \lstinputlisting[linerange=calc_square_3-calc_square_4]{1_resister_network.cpp}
}
首先计算$R_\text{ab}$。
先使用\textbf{基于LDL分解的直接解法}。
定义半带状矩阵\verb|l|与数组\verb|d|存储分解结果。

LDL分解的具体实现 \verb|ldl_factor()| 会在\autoref{ssec:ldl} 中详述。
总之,该分解将对称正定矩阵$\vb{A}$分解为
\begin{equation}\label{eq:ldl}
    \vb{A} = \vb{L}\vb{D}\vb{L}^T,
\end{equation}
那么矩阵方程$\vb{A}\vb*{x}=\vb*{b}$就变成了
\begin{equation}\label{eq:after_ldl}
    \vb{L}\qty(\vb{D}\qty(\vb{L}^T\vb*{x})) = \vb*{b},
\end{equation}
可用三角矩阵的回代算法求解。
{
    \linespread{1.0}
    \lstinputlisting[linerange=calc_square_4-calc_square_5]{1_resister_network.cpp}
}
生成非齐次项$\vb*{b}$.
先将其初始化为0矢量。
获得b点$(N,0)$的编号,然后将$\vb*{b}$的该分量设为1,表明流入单位电流。
{
    \linespread{1.0}
    \lstinputlisting[linerange=calc_square_5-calc_square_6]{1_resister_network.cpp}
}
计算并输出,通过三次回代将式 \eqref{eq:after_ldl} 中的三个简单矩阵依次消去。
由于将a点电势设为0,b点的电势在单位电流的时候即为$R_\text{ab}$。
{
    \linespread{1.0}
    \lstinputlisting[linerange=calc_square_6-calc_square_7]{1_resister_network.cpp}
}
再使用\textbf{基于共轭梯度法的迭代解法}。
函数 \verb|conj_grad()| 使用返回值报告是否成功求解。
共轭梯度法的具体实现会在\autoref{ssec:conj_grad}中详述。
{
    \linespread{1.0}
    \lstinputlisting[linerange=calc_square_7-end:calc_square]{1_resister_network.cpp}
}
后面对$R_\text{ac}$的求解与前面求解$R_\text{ab}$步骤几乎完全一致,故不再更多解释。

\subsection{三角型网格\texttt{calc\_triangle<N>()}}
与\autoref{ssec:calc_square} 几乎完全一致,唯一的区别是流入电流的位置不同:
{
    \linespread{1.0}
    \lstinputlisting[linerange=beg:calc_triangle-end:calc_triangle]{1_resister_network.cpp}
}

\subsection{六边形型网格\texttt{calc\_hex<N>()}}
与\autoref{ssec:calc_square} 几乎完全一致,唯一的区别是流入电流的位置\footnote{这里其实使用了与b点等价的另一点。亦即\autoref{fig:abs_hex_network} 中的14而非17。}不同:
{
    \linespread{1.0}
    \lstinputlisting[linerange=beg:calc_hex-end:calc_hex]{1_resister_network.cpp}
}

\subsection{LDL分解\texttt{ldl\_factor()}}
\label{ssec:ldl}
LDL分解
\begin{equation}\tag{\ref{eq:ldl}}
    \vb{A} = \vb{L}\vb{D}\vb{L}^T.
\end{equation}
令$\vb{T} = \vb{D}\vb{L}^T$,则$\vb{T}$的对角元即为$\vb{D}$。
它们的元素满足
\begin{equation}
    t_{ji} = a_{ij} - \sum_{k=0}^{j-1} l_{ik}t_{kj},\quad
    l_{ij} = \frac{t_{ji}}{d_{j}},\quad
    d_{i} = a_{ii} - \sum_{k=0}^{i-1} l_{ik}t_{ki}.
\end{equation}
按$d_0,l_{10},d_1,l_{20},l_{21},d_2,l_{30},l_{31},l_{32},d_3,\cdots$的顺序求解即可。

对于带状矩阵,可以通过避免不必要的求和来减少运算。
下面给出带状对称矩阵的LDL分解运算。
(位于\texttt{misc/linear\_eq\_direct.h}中)
{
    \linespread{1.0}
    \lstinputlisting[linerange=beg:ldl_factor-end:ldl_factor]{../misc/linear_eq_direct.h}
}
\verb|temp_t|存储矩阵$\vb{T}$;
\verb|limit|存储每个子循环中$j$的求和下界——0与$i-\text{texttt{M}}$中的大者;
\verb|t_jj|则是为了减少不必要的成员访问。

此外,针对对角元$d_j=0$的情况,只需将$\vb{L}$的对应列取为0即可。

\subsection{共轭梯度法\texttt{conj\_grad()}}
\label{ssec:conj_grad}
函数接口如下。
(位于\texttt{misc/linear\_eq\_iterative.h}中)
默认在残差与初始残差的范数之比为\texttt{rel\_epsilon}$=1\times 10^{-15}$时停止迭代,并返回0。
若到\texttt{max\_times}$=1000$仍没能停止,返回1。
此外,\texttt{sparse}参数控制是否将$\vb{A}$视作稀疏矩阵。
{
    \linespread{1.0}
    \lstinputlisting[linerange=beg:conj_grad_dec-end:conj_grad_dec]{../misc/linear_eq_iterative.h}
}
共轭梯度法的基本想法是不断地确定$\vb*{x}$的搜索方向$\vb*{p}$,并使残差$\vb*{r}=\vb*{b}-\vb{A}\vb*{x}$在此方向上取最小。
随后,在$\vb*{p}$与$\vb*{r}$张成的平面上寻找新的$\vb*{p}$。
其利用向量正交性质优化后的伪代码见\autoref{fig:ldl}。

\begin{figure}
    \centering
    \includegraphics[width=0.7\textwidth,page=93]{../chap_3_net.pdf}
    \caption{LDL分解的伪代码}
    \label{fig:ldl}
\end{figure}

实现如下
{
    \linespread{1.0}
    \lstinputlisting[linerange=beg:conj_grad_imp-end:conj_grad_imp]{../misc/linear_eq_iterative.h}
}
其中,为减少运算,使用\verb|Ap|暂存$\vb{A}\vb*{p}$的结果;
使用\verb|rr|暂存$\vb*{r}^T\vb*{r}$的结果。

\section{计算结果}
将各计算结果总结于\autoref{tab:R_ab_result}。
使用的数据类型为\texttt{double},其相对精度大约至$10^{-16}$,故在输出时,将最大有效位数设置为了16。

\begin{table}
\centering
\caption{各网格的电阻计算结果}
\label{tab:R_ab_result}
\begin{tabular}{ccll}
    \toprule
    & \texttt{N} & \multicolumn{1}{c}{直接解法} & \multicolumn{1}{c}{迭代解法} \\
    \midrule
    \multirow{4}{*}{方型$R_\text{ab}$}   & 1          & 0.75                     & 0.75                     \\
     & 4  & 1.901515151515153 & 1.901515151515151 \\
     & 16 & 3.463587937288163 & 3.463587937288109 \\
     & 64 & 5.171646779478962 & 5.171646779479373 \\
    \hline
    \multirow{4}{*}{方型$R_\text{ac}$}   & 1          & 1                        & 0.9999999999999999       \\
     & 4  & 2.136363636363636 & 2.136363636363637 \\
     & 16 & 3.685592463430818 & 3.685592463430749 \\
     & 64 & 5.392376786288083 & 5.39237678628829  \\
    \hline
    \multirow{4}{*}{三角$R_\text{ab}$}  & 1          & 0.6666666666666666       & 0.6666666666666666       \\
     & 4  & 1.67479674796748  & 1.67479674796748  \\
     & 16 & 3.024372524685346 & 3.024372524685373 \\
     & 64 & 4.503230022407012 & 4.503230022407678 \\
    \hline
    \multirow{3}{*}{六边形$R_\text{ab}$} & 4          & 2.819373942470396        & 2.819373942470387        \\
     & 16 & 6.534376528387103 & 6.534376528387018 \\
     & 64 & 10.83543674481138 & 10.83543674481063 \\
    \bottomrule
\end{tabular}
\end{table}

可以发现,两种结果十分接近,但在最后几位略有差别。
这反映了浮点数运算过程中引入的误差。
同时,\texttt{N}越大,这一偏差也相应增大。
这与矩阵规模越大,进行的运算次数越多是有关的。

\subsection{耗时情况与分析}
主函数\verb|main()|如下
{
    \linespread{1.0}
    \lstinputlisting[linerange=beg:main-end:main]{1_resister_network.cpp}
}
测试得到耗时约为2.7~s。
在使用编译参数~\texttt{-O3}进行优化且接通电源时,耗时约为0.9~s。

下面考察两种方法的耗时差别。
在\verb|calc_square()|的基础上定义计时测试函数\verb|timing()|如下,并取消\verb|main()|结尾部分的注释进行计时测试。
使用的时钟为\verb|<chrono>|中的单向时钟\verb|chrono::steady_clock|。
{
    \linespread{1.0}
    \lstinputlisting[linerange=beg:timing-end:timing]{1_resister_network.cpp}
}
计时结果如\autoref{tab:timing} 所示。
计算单次运行用时,用双对数坐标绘制于\autoref{fig:timing}。
图中也展示了进行幂函数拟合后的结果。

\begin{table}
\centering
\caption{对方型网格的$R_\text{ab}$用不同方式进行求解的用时测量}
\label{tab:timing}
\begin{tabular}{ccP{4.0}P{5.0}}
    \toprule
    \texttt{N} & \texttt{repeat} & \multicolumn{1}{c}{直接解法 (ms)}& \multicolumn{1}{c}{迭代解法 (ms)} \\ \midrule
    1 & $10^6$ & 815       & 2888      \\
    2 & $10^6$ & 3352      & 14321     \\
    4 & $10^4$ & 218       & 816       \\
    8 & $10^4$ & 1595      & 6664      \\
    16& $10^3$ & 1604      & 4824      \\
    64& $10^1$ & 2764    & 5158      \\
    128& $2$ & 8531    & 16996      \\ \bottomrule
\end{tabular}
\end{table}

\begin{figure}
    \centering
    \includegraphics[width=0.95\textwidth]{figures/timing.pdf}
    \caption{对方型网格的$R_\text{ab}$用不同方式进行求解的用时测量}
    \label{fig:timing}
\end{figure}

容易发现,在这些测试用例下,直接解法均比迭代解法快速;但是,迭代解法的复杂度约为$O(n^{3.05})$,是优于直接解法的$O(n^{3.22})$的。
根据拟合结果,可以推测在\texttt{N} $\approx 6000$时会发生反超。
但可惜的是,\texttt{N} $= 256$时本人计算机的内存便不足以支撑运算,无法验证这一点。

那么,迭代解法用时偏长是否可能与我将终止判据(相对误差小于$10^{-15}$)设得过小有关呢?
我对此进行实验,将判据改为$10^{-8}$,但迭代解法用时并没有显著缩短。

\textbf{总结:}\begin{inparaenum}
    \item 迭代解法复杂度低于直接解法;
    \item 但迭代解法在小规模数据下不如直接解法快。
\end{inparaenum}

\section{三角\texorpdfstring{$RLC$}{RLC}网格的频率响应}
三角网格的交流电情形基本基于\autoref{ssec:triangle},但仍使用了几个不同的函数,如下
{
    \linespread{1.0}
    \lstinputlisting[linerange=beg:ac_api-end:ac_api]{1_resister_network.cpp}
}
定义了枚举类\verb|Ac_Type|以增加可读性;此外针对复数做了一些改动。

\subsection{交流网格的实现}
\paragraph{\texttt{\_triangle\_ac\_connections()}}
的返回值不再是一个\verb|vector|,而是同时保存了连线的空间信息与元件信息的\texttt{map<size\_t, AC\_Type>},以便提示\texttt{triangle\_ac\_network()}生成什么样的元件。函数定义如下。
{
    \linespread{1.0}
    \lstinputlisting[linerange=beg:ac_conn-end:ac_conn]{1_resister_network.cpp}
}

\paragraph{\texttt{triangle\_ac\_network()}}
再利用\texttt{\_triangle\_ac\_connections()}生成网格。
该函数整体上与 \nameref{para:tri_net} 十分接近,定义如下:
{
    \linespread{1.0}
    \lstinputlisting[linerange=beg:ac_net-end:ac_net]{1_resister_network.cpp}
}
主要区别为:
\begin{compactenum}
    \item 提供参数\verb|hermite|以生成厄密共轭的系数矩阵;
    \item 根据\texttt{\_triangle\_ac\_connections()}返回的\texttt{AC\_Type}的值确定电容、电感还是电阻;
    \item 接收参数\texttt{omega}作为交流电的圆频率。
\end{compactenum}

\subsection{响应曲线的计算与结果}
\paragraph{\texttt{calc\_triangle\_ac()}}
的实现与之前有所不同。

交流电下产生的系数矩阵是一个对称复矩阵,而复的LDL分解要求一个厄密矩阵,且算法与实数情况有较大不同。
鉴于要求的矩阵规模较小,故直接解法使用高斯约当消元法求逆。

共轭梯度法迭代也要求厄密矩阵,故使用$\vb{G}^\dagger\vb{G}$替代$\vb{G}$,使用$\vb{G}^\dagger\vb*{b}$替代$\vb*{b}$。
同时,共轭梯度法中的所有转置运算都被替换为求厄密共轭。

函数定义如下:
{
    \linespread{1.0}
    \lstinputlisting[linerange=beg:ac_calc-end:ac_calc]{1_resister_network.cpp}
}

直接解法与迭代解法的结果十分接近,故仅取直接解法的结果列于\autoref{tab:ac} 中。
可以发现,在$\omega=1$附近,虚部的符号发生了改变。
绘制图像于\autoref{fig:ac_res} 中。

\begin{longtable}{cll}
    \caption{三角型网格在不同的角频率$\omega$下的阻抗}
    \label{tab:ac}\\
    \toprule
    $\omega$ & \multicolumn{1}{c}{实部}& \multicolumn{1}{c}{虚部} \\
    \midrule
    \endfirsthead
    \multicolumn{3}{l}{接上页} \\
    \toprule
    $\omega$ & \multicolumn{1}{c}{实部}& \multicolumn{1}{c}{虚部} \\
    \midrule
    \endhead
    \bottomrule
    \multicolumn{3}{r}{接下页} \\
    \endfoot
    \bottomrule
    \endlastfoot
    0.05&	2.086570517377066&  \multicolumn{1}{@{$-$}l}{0.03697452355659513}\\
    0.15&	2.105590723198735&  \multicolumn{1}{@{$-$}l}{0.1292639140107299}\\
    0.25&	2.116312012793758&  \multicolumn{1}{@{$-$}l}{0.2613200206792055}\\
    0.35&	2.089046293761019&  \multicolumn{1}{@{$-$}l}{0.4314963859144606}\\
    0.45&	2.000152657164885&  \multicolumn{1}{@{$-$}l}{0.6294205602210458}\\
    0.55&	1.820938280388908&  \multicolumn{1}{@{$-$}l}{0.8395385724838142}\\
    0.65&	1.511673777909439&  \multicolumn{1}{@{$-$}l}{1.019674292104832}\\
    0.75&	1.05922940811744&   \multicolumn{1}{@{$-$}l}{1.072184275825388}\\
    0.85&	0.5393755324216926& \multicolumn{1}{@{$-$}l}{0.8886779197097765}\\
    0.95&	0.08562948250802678&    \multicolumn{1}{@{$-$}l}{0.386328021169947}\\
    1.05&	0.07809376057427178&    0.3694391489922765\\
    1.15&	0.439752147816773&  0.819426665811231\\
    1.25&	0.8002429070290147& 1.014324319972504\\
    1.35&	1.105754417601758&  1.075977616913043\\
    1.45&	1.347357192295445&  1.062758845386196\\
    1.55&	1.530122196546158&  1.012898907626556\\
    1.65&	1.665960840831235&  0.9491265287520781\\
    1.75&	1.766970639544693&  0.8830669088315197\\
    1.85&	1.842816453039244&  0.8199255073122849\\
    1.95&	1.900520408503763&  0.7616633450148784\\
    2.05&	1.945013936696054&  0.7087376809319406\\
    2.15&	1.979742895951405&  0.6609617168771185\\
    2.25&	2.007136475411896&  0.617905830610615\\
    2.35&	2.028932965855138&  0.579077196060318\\
    2.45&	2.046397654716548&  0.5439959964838148\\
    2.55&	2.060467405063928&  0.5122244130065424\\
    2.65&	2.071847178436657&  0.4833745621845713\\
    2.75&	2.081075313834781&  0.4571074652310582\\
    2.85&	2.088568387232175&  0.4331285753546922\\
    2.95&	2.094652567441202&  0.4111823425027696\\
    3.05&	2.099585907409738&  0.3910468793872911\\
    3.15&	2.103574451311726&  0.3725291393382476\\
    3.25&	2.106784052592502&  0.355460720822583\\
    3.35&	2.109349169132848&  0.3396942857754414\\
    3.45&	2.111379494858376&  0.3251005295764186\\
    3.55&	2.112965020158815&  0.3115656269562989\\
    3.65&	2.114179935671448&  0.2989890800790772\\
    3.75&	2.115085673771645&  0.2872819029663876\\
    3.85&	2.115733299647044&  0.276365085933262\\
    3.95&	2.116165406453527&  0.2661682928825249\\
    4.05&	2.116417628584686&  0.2566287524356488\\
    4.15&	2.116519858176469&  0.2476903107786637\\
    4.25&	2.116497229061063&  0.2393026198277187\\
    4.35&	2.116370917088055&  0.231420439012023\\
    4.45&	2.116158794418082&  0.2240030327959009\\
    4.55&	2.115875966941544&  0.2170136491698679\\
    4.65&	2.115535217598628&  0.2104190668675873\\
    4.75&	2.115147373523699&  0.2041892011231947\\
    4.85&	2.114721611212198&  0.1982967594631869\\
    4.95&	2.114265711027013&  0.1927169404031023\\
\end{longtable}

\begin{figure}
    \centering
    \includegraphics[width=.95\textwidth]{figures/ac_res.pdf}
    \caption{交流电网格的响应曲线}
    \label{fig:ac_res}
\end{figure}

观察\autoref{fig:ac_res},可以很明显地发现,这一网格存在一个共振频率:$\omega=1$。

\section{总结}
本题中,本人使用线性方程组的直接解法与迭代解法求解了\autoref{fig:network} 所示的多种不同规模的电阻网格的电阻,结果列于\autoref{tab:R_ab_result} 中。
对两种方法的耗时以及复杂度做了简要分析,反映于\autoref{fig:timing}、\autoref{tab:timing} 中。

此外,对含交流元件的三角网格的频率响应做了计算,结果见\autoref{fig:ac_res}、\autoref{tab:ac}。


\chapter{Logistic模型}
\lstset{
    language=Python,
    rangeprefix=\#\ ,
    rangesuffix=\ \#,
}
迭代关系
\begin{equation}
    x_{n+1} = f(x_n) = rx_n\qty(1-x_n)
\end{equation}
定义了Logistic模型,其中的$f(x)$被称为Logistic函数。
本题将研究该序列聚点的分形与震荡行为随可调参数$r$的变化。

下面将结合程序 \verb|2_logistic.py|,依题目要求进行说明。

定义函数 \verb|g()|,为$f(x)$中除了系数$r$之外的部分。
{
    \linespread{1.0}
    \lstinputlisting[linerange=gx-end:gx]{2_logistic.py}
}
定义生成器函数 \verb|logistic_generator()|,以方便高效地生成序列。
{
    \linespread{1.0}
    \lstinputlisting[linerange=log_gen-end:log_gen]{2_logistic.py}
}

\section{单聚点情况}\label{sec:single}
首先取$r=0.5, 1.5$,观察序列$\{x_n\}$的迭代情况。
有关代码如下。
各个$r$值均使用随机生成的10个$x_0$作为初始值。
{
    \linespread{1.0}
    \lstinputlisting[linerange=problem_1-end:problem_1]{2_logistic.py}
}

生成的数据图像见\autoref{fig:problem_1}。
可以看到,\hyperref[fig:problem_1_0.5]{$r=0.5$} 时,序列极快地收敛到0;
\hyperref[fig:problem_1_1.5]{$r=1.5$} 时,序列也较快地收敛到$\frac{1}{3}$。

\begin{figure}
    \centering
    \subfloat[$r=0.5$]{
        \label{fig:problem_1_0.5}
        \includegraphics[width=0.48\textwidth]{figures/p1_0.5.pdf}
    }
    \subfloat[$r=1.5$]{
        \label{fig:problem_1_1.5}
        \includegraphics[width=0.48\textwidth]{figures/p1_1.5.pdf}
    }
    \caption{$r=0.5, 1.5$时Logistic函数的迭代行为}
    \label{fig:problem_1}
\end{figure}

\section{单聚点收敛分析}
若序列$\{x_n\}$仅有一个聚点$x^*$,那么$x^*$一定有
\begin{equation}
    x^* = f(x^*) = rx^*(1-x^*).
\end{equation}
显然,两根分别为$0$与$1-\frac{1}{r}$。

\subsection{收敛条件}
作为聚点,其一定满足稳定条件——亦即,相对于该点的小偏移并不会导致序列发散。
或者说,偏差$x_n-x^*$应随着$n$的增大而减小。
注意到,偏差
\begin{equation}
    x_{n+1} - x^* = f(x_n) - f(x^*),
\end{equation}
那么前后的误差之比为
\begin{equation}\label{eq:error}
    \qty|\frac{x_{n+1} - x^*}{x_{n} - x^*}|
    = \qty|\frac{f(x_n) - f(x^*)}{x_{n} - x^*}|.
\end{equation}

如果序列收敛,那么该比例定在$x_n\to x^*$时不大于1。
考虑导数的定义,那么有
\begin{equation}
    \qty|f'(x^*)| \le 1.
\end{equation}
此即序列收敛到$x^*$的必要条件。

\subsection{计算聚点\texorpdfstring{$x^*$}{x*}与\texorpdfstring{$r$}{r}的关系}
易得,$f'(x) = r(1-2x)$。
代入两根$0$与$1-\frac{1}{r}$,得到$r$与$2-r$。
由于聚点处那么,
\begin{equation}\label{eq:prob_2_res}
    x^* = \begin{cases}
        r,& 0\le r\le 1,\\
        2-r,&   1<r\le 3.
    \end{cases}
\end{equation}

使用如下程序数值计算不同$r$的聚点
{
    \linespread{1.0}
    \lstinputlisting[linerange=problem_2-end:problem_2]{2_logistic.py}
}
\verb|N| 为在$r$单位长度上取多少数据点。
认为收敛较快,将$x_{1000}$作为聚点使用。
数据图见\autoref{fig:problem_2}。
可以看到,这与式 \eqref{eq:prob_2_res} 中给出的解析表达是一致的。

\begin{figure}
    \centering
    \includegraphics[width=0.9\textwidth]{figures/p2_2.98.pdf}
    \caption{$0\le r\le 3$时的序列收敛点$x^*$}
    \label{fig:problem_2}
\end{figure}

\subsection{收敛阶与收敛速度}
由式 \eqref{eq:error} 可以看到,
\begin{equation}\label{eq:conv_speed}
    \lim_{n\to\infty} \qty|\frac{x_{n+1} - x^*}{x_{n} - x^*}| = \qty|f'(x^*)|.
\end{equation}

根据收敛阶与收敛速度的定义,可以得到,收敛阶为$1$,收敛速度$s = -\log_{10}\qty|f'(x^*)|$。

\section{双聚点情况}
上一节中在式 \eqref{eq:prob_2_res} 已经得到在$r>r_1=3$时两个根均无法成为聚点。
取$r=r_1+0.1=3.1$,随机生成三个$x_0$,观察序列情况。
程序如下,数据图见\autoref{fig:problem_3}。
{
    \linespread{1.0}
    \lstinputlisting[linerange=problem_3-end:problem_3]{2_logistic.py}
}

\begin{figure}
    \centering
    \includegraphics[width=0.9\textwidth]{figures/p3_3.1.pdf}
    \caption{$r = 3.1$时的Logistic序列}
    \label{fig:problem_3}
\end{figure}

容易发现,序列出现了周期为2的震荡行为。
两个聚点分别为0.764与0.558。

\section{双聚点收敛分析}\label{sec:double_conv}
序列$\{x_n\}$出现了震荡行为,周期为2。
那么,任何一个聚点$x^*$都应有
\begin{equation}
    x^* = f(f(x^*)) = r^2x^*(1-x^*)\qty(1-rx^*(1-x^*)).
\end{equation}
显然,该方程有4个根。

这些根中的聚点一定满足稳定条件。
考虑其中一个子序列$\{x_{2k}\}$,偏差$x_{2k}-x_1^*$应随着$k$的增大而减小。

注意到,偏差
\begin{equation}
    x_{2k+2} - x_1^* = f(f(x_{2k})) - f(f(x_1^*)),
\end{equation}
那么前后的偏差之比为
\begin{equation}\label{eq:error_double}
    \qty|\frac{x_{2k+2} - x_1^*}{x_{2k} - x_1^*}|
    = \qty|\frac{f(f(x_{2k})) - f(f(x_1^*))}{x_{2k} - x_1^*}|.
\end{equation}

如果子序列收敛,那么该比例定在$x_{2k}\to x_1^*$时不大于1。
考虑导数的定义,那么有
\begin{equation}
    \qty|\dv{f(f(x))}{x}|_{x=x_1^*} \le 1.
\end{equation}
根据链式法则,以及$x_1^* = f(x_2^*)$,$x_2^* = f(x_1^*)$,上式即
\begin{equation}
    \qty|\dv{f(u)}{u}_{u=x_2^*}|\qty|\dv{f(x)}{x}_{x=x_1^*}| = \qty|f'(x_1^*)f'(x_2^*)| \le 1.
\end{equation}
此即子序列收敛到$x_1^*$的必要条件。
注意到,此式中$x_1^*$与$x_2^*$对称,故另一聚点的收敛条件也为此。

使用如下程序数值计算不同$r$的聚点
{
    \linespread{1.0}
    \lstinputlisting[linerange=problem_4-end:problem_4]{2_logistic.py}
}
\verb|N| 为在$r$单位长度上取多少数据点。
认为收敛较快,将$x_{1000}$附近的两个数据作为聚点使用。
使用了\textsf{Python}的集合数据类型以消除重复。
数据图见\autoref{fig:problem_4}。

\begin{figure}
    \centering
    \includegraphics[width=0.9\textwidth]{figures/p4_3.43.pdf}
    \caption{二周期震荡的序列收敛点$x^*$}
    \label{fig:problem_4}
\end{figure}

\section{更大的振荡周期以及收敛速度}
\subsection{振荡周期增加}
绘制\autoref{fig:problem_4} 时,已经发现在$r\gtrsim 3.44$时,周期会不再是二。
下面取$r=3.5, 3.55$,给出周期为4与8的序列。
为避免图片杂乱,仅使用了一个随机的$x_0$,且不再添加连线。
代码如下,数据图见\autoref{fig:problem_5_cycle}。
{
    \linespread{1.0}
    \lstinputlisting[linerange=problem_5-mid:problem_5]{2_logistic.py}
}

\begin{figure}
    \centering
    \subfloat[$r=3.5$,周期为4]{
        \label{fig:problem_5_3.5}
        \includegraphics[width=0.48\textwidth]{figures/p5_3.5.pdf}
    }
    \subfloat[$r=3.55$,周期为8]{
        \label{fig:problem_5_3.55}
        \includegraphics[width=0.48\textwidth]{figures/p5_3.55.pdf}
    }
    \caption{$r=3.5, 3.55$时Logistic函数的迭代行为}
    \label{fig:problem_5_cycle}
\end{figure}

\subsection{描述收敛速度}
根据 \autoref{sec:double_conv} 推广式 \eqref{eq:conv_speed},容易得到在振荡周期为$T$时,任一个收敛子序列满足
\begin{equation}
    \lim_{n\to\infty} \qty|\frac{x_{n+T} - x_i^*}{x_{n} - x_i^*}| = \qty|f'(x_1^*)f'(x_2^*)\cdots f'(x_T^*)|,
\end{equation}
其中$x_i^*$为收敛到的第$i$个聚点。
可以发现,等式右边的比例系数(记为$C(r)$)是对$r$相同的所有收敛子序列都一致的。

虽然正式定义的收敛速度为$-\log_{10} C(r)$,但仍可把$C(r)$直接作为收敛速度的度量。
下面定义的函数 \verb|converge_speed()| 能够对给定的$r$估算收敛系数$C(r)$。
{
    \linespread{1.0}
    \lstinputlisting[linerange=converge_speed-end:converge_speed]{2_logistic.py}
}
函数分为四部分。
\texttt{skip}参数决定从$x_\text{\texttt{skip}}$开始计算收敛速率(因为太前面的$x_n$变化过于剧烈)。
\begin{compactitem}
    \item[计算周期$T$] 生成前$n$(默认为30000)个数据,截取最后512个(周期更大的忽略)转换为\texttt{set}。利用该数据类型元素互不重复性质得到周期。
    \item[计算$x^*$] 此应为与$x_\text{\texttt{skip}}$在同一子序列中的$n$最大的$x_n$。
    \item[计算偏差] 不断计算子序列中的$x_n$与$x^*$的偏差,并计算该偏差减小的比例系数。若碰到偏差为0的情况,跳出循环。
    \item[计算速率平均值] 若得到大于1的速率,说明发散,仍返回1;若结果列表为空,说明一开始偏差即为0,即收敛极快,返回0。
\end{compactitem}

在区间$[0,4)$上取了400个点计算收敛速率,代码如下
{
    \linespread{1.0}
    \lstinputlisting[linerange=mid:problem_5-end:problem_5]{2_logistic.py}
}
结果见\autoref{fig:conv_speed}。
可以发现,
\begin{compactenum}
    \item 该曲线大致上呈周期逐步减小的锯齿状。
    \item $C(r)=1$的位置与振荡周期的分叉点是一致的,说明此处收敛极慢。
    \item 在$r\gtrsim 3.6$时,几乎恒发散。
\end{compactenum}

\begin{figure}
    \centering
    \includegraphics[width=.8\textwidth]{figures/p5_speed_line.pdf}
    \caption{收敛系数$C(r)$}
    \label{fig:conv_speed}
\end{figure}

\section{后续\texorpdfstring{$x^*$}{x*}的震荡}
绘制完整的\autoref{fig:problem_4} 于\autoref{fig:problem_6_full}。
将绘图区的左下边界依次设为各分叉点,绘图于\autoref{fig:problem_6_right}。

\begin{figure}
    \centering
    \includegraphics[width=0.8\textwidth]{figures/p6_4_0.pdf}
    \caption{序列收敛点$x^*$随$r$的关系}
    \label{fig:problem_6_full}
\end{figure}

\begin{figure}
    \centering
    \subfloat[周期1]{
        \label{fig:problem_6_right_0}
        \includegraphics[width=0.48\textwidth]{figures/p6_4_1.pdf}
    }
    \subfloat[周期2]{
        \label{fig:problem_6_right_1}
        \includegraphics[width=0.48\textwidth]{figures/p6_4_2.pdf}
    }
    \\
    \subfloat[周期4]{
        \label{fig:problem_6_right_2}
        \includegraphics[width=0.48\textwidth]{figures/p6_4_3.pdf}
    }
    \subfloat[周期8]{
        \label{fig:problem_6_right_3}
        \includegraphics[width=0.48\textwidth]{figures/p6_4_4.pdf}
    }
    \caption{序列收敛点$x^*$随$r$的关系,左下角取为各分叉点。子图标题中的周期指图片中最左侧的那条分支的周期数}
    \label{fig:problem_6_right}
\end{figure}

可以发现,这些数据图呈现出自相似行为,并且局限在$x^*<\frac{r}{4}$的区域。
但是,这些规律的分叉线占全图的比例越来越小——原因应该是无穷周期分叉点并不是在$r=4$,而是在$r_\infty \approx 3.57$附近。

因此,将绘图区的右边界设为$r_\infty$(使用\texttt{matplotlib}的绘图窗口测量,为3.56994592),重新绘图于\autoref{fig:problem_6_part}。
这种绘图方式使自相似十分明显。

\begin{figure}
    \centering
    \subfloat[周期1]{
        \label{fig:problem_6_part_0}
        \includegraphics[width=0.31\textwidth]{figures/p6_3.56994592_1.pdf}
    }
    \subfloat[周期2]{
        \label{fig:problem_6_part_1}
        \includegraphics[width=0.31\textwidth]{figures/p6_3.56994592_2.pdf}
    }
    \subfloat[周期4]{
        \label{fig:problem_6_part_2}
        \includegraphics[width=0.31\textwidth]{figures/p6_3.56994592_3.pdf}
    }
    \\
    \subfloat[周期8]{
        \label{fig:problem_6_part_3}
        \includegraphics[width=0.31\textwidth]{figures/p6_3.56994592_4.pdf}
    }
    \subfloat[周期16]{
        \label{fig:problem_6_part_4}
        \includegraphics[width=0.31\textwidth]{figures/p6_3.56994592_5.pdf}
    }
    \subfloat[周期32]{
        \label{fig:problem_6_part_5}
        \includegraphics[width=0.31\textwidth]{figures/p6_3.56994592_6.pdf}
    }
    \\
    \subfloat[周期64]{
        \label{fig:problem_6_part_6}
        \includegraphics[width=0.31\textwidth]{figures/p6_3.56994592_7.pdf}
    }
    \subfloat[周期128]{
        \label{fig:problem_6_part_7}
        \includegraphics[width=0.31\textwidth]{figures/p6_3.56994592_8.pdf}
    }
    \caption{序列收敛点$x^*$随$r$的关系,左下角取为各分叉点。子图标题中的周期指图片中最左侧的那条分支的周期数。仅绘制了自相似部分}
    \label{fig:problem_6_part}
\end{figure}

\section{自相似行为}
下面考察上一节中给出的自相似行为。
对各分叉点的$r$进行测量,结果如\autoref{tab:r_and_F} 所示。
相邻分叉点的间距$\Delta r$近似以比例4.6缩小。
通过不断放大绘图,得到$r_\infty \approx 3.56994592$。

\begin{table}
    \centering
    \caption{各分叉点处的$r$值}
    \label{tab:r_and_F}
    \begin{tabular}{lll}
    \toprule
    \multicolumn{1}{c}{$r$} & \multicolumn{1}{c}{$\Delta r$} & \multicolumn{1}{c}{$F$} \\ \midrule
    1.000 & 2.000 & 4.45 \\
    3.000 & 0.449 & 4.73 \\
    3.449 & 0.0949 & 4.65 \\
    3.5439 & 0.0204 & 4.59 \\
    3.5643 & 0.00444 & 4.70 \\
    3.56874 & 0.000944 & 4.62 \\
    3.569684 & 0.000204 & 4.49 \\
    3.569888 & 0.0000454 & 4.87 \\
    3.5699334 & 0.00000931 &  \\
    3.56994271 & \multicolumn{1}{r}{平均:}& 4.6 \\ \bottomrule
    \end{tabular}
\end{table}

查阅资料发现,这实际上形成了Logistic map\footnote{\url{https://en.wikipedia.org/w/index.php?title=Logistic_map&oldid=946693424}}。
相邻$\Delta r$之比的极限为Feigenbaum constant $\delta\approx 4.66920$,$r_\infty \approx 3.56995$。
可以发现,我对$r_\infty$的测量是比较精确的。

\section{\texorpdfstring{$r=4$}{r=4}时的解析解}
$r=4$时,序列为
\begin{equation}
    x_{n+1} = f(x_n) = 4x_n(1-x_n).
\end{equation}
记$x_n = \sin^2 y_n$,则
\begin{equation}
    \sin^2 y_{n+1} = 4\sin^2 y_n\cos^2 y_n = \sin[2](2y_n).
\end{equation}

不妨让$y_n \in [0, \pi]$,那么
\begin{equation}\label{eq:y_at_4}
    y_{n+1} = 2y_n \bmod \pi.
\end{equation}
由于2为有理数,而$\pi$为无理数,式 \eqref{eq:y_at_4} 几乎不可能获得周期性,除非$y_0$为$\pi$的倍数。
既然$\{y_n\}$无周期性,$\{x_n\}$也无周期性。

\section{使用其它函数}
选取$g(x) = \sin(\pi x)$,将$f(x)$替换为$rg(x)$重新进行前述计算。
程序文件见 \verb|2_logistic_sin.py|。
由于程序本身十分接近,故不再引用代码进行解释。

\subsection{单聚点情况}
前面 \autoref{sec:single} 中选取的$r=0.5, 1.5$分别对应单聚点时方程$x=f(x)$的两个根。
在$g(x) = \sin(\pi x)$时,选取对应点为0.1与0.5。
序列行为见\autoref{fig:problem_9.1}。

\begin{figure}
    \centering
    \subfloat[$r=0.1$]{
        \label{fig:problem_9.1_0.15}
        \includegraphics[width=0.48\textwidth]{figures/p9.1_0.1.pdf}
    }
    \subfloat[$r=0.5$]{
        \label{fig:problem_9.1_0.5}
        \includegraphics[width=0.48\textwidth]{figures/p9.1_0.5.pdf}
    }
    \caption{$r=0.1, 0.5$时$r\sin(\pi x)$函数的迭代行为}
    \label{fig:problem_9.1}
\end{figure}

可以看到,$r=0.1$时序列收敛到0,$r=0.5$时,序列收敛到0.5附近。
绘制单聚点时聚点$x^*$随$r$的关系于\autoref{fig:problem_9.2},与\autoref{fig:problem_2} 十分相似。

\begin{figure}
    \centering
    \includegraphics[width=0.9\textwidth]{figures/p9.2_0.7133333333333334.pdf}
    \caption{$0\le r\le 0.72$时的序列收敛点$x^*$}
    \label{fig:problem_9.2}
\end{figure}

根据单聚点条件$|f'(x^*)| \le 1$,可得到单聚点的$r$值上限$r_1$满足:
\begin{equation}
    r_1=-\frac{1}{\pi\cos u},\quad \text{其中 }u+\tan u=0.
\end{equation}
近似解得$r_1=0.71996$。

\subsection{双聚点情况}
前面得到,在$r>r_1=0.71996$时,序列无法收敛到单值。
取$r=0.82$,随机选取几个初值计算序列,绘制于\autoref{fig:problem_9.3}。
序列也出现了周期为 2 的震荡行为。

\begin{figure}
    \begin{minipage}{0.48\textwidth}
        \centering
        \includegraphics[width=\textwidth]{figures/p9.3_0.82.pdf}
        \caption{$r = 0.82$时的双聚点序列}
        \label{fig:problem_9.3}
    \end{minipage}
    \hspace{2pt}
    \begin{minipage}{0.48\textwidth}
        \centering
        \includegraphics[width=\textwidth]{figures/p9.4_0.82.pdf}
        \caption{二周期震荡的序列收敛点$x^*$}
        \label{fig:problem_9.4}
    \end{minipage}
\end{figure}

补充双周期震荡时的聚点于\autoref{fig:problem_9.4}。

\subsection{更大的振荡周期以及收敛速度}
取$r=0.84, 0.86$,序列震荡周期变为4与8。
(见\autoref{fig:problem_9.5_cycle})

\begin{figure}
    \centering
    \subfloat[$r=3.5$,周期为4]{
        \label{fig:problem_9.5_0.84}
        \includegraphics[width=0.48\textwidth]{figures/p9.5_0.84.pdf}
    }
    \subfloat[$r=3.55$,周期为8]{
        \label{fig:problem_9.5_0.86}
        \includegraphics[width=0.48\textwidth]{figures/p9.5_0.86.pdf}
    }
    \caption{$r=3.5, 3.55$时Logistic函数的迭代行为}
    \label{fig:problem_9.5_cycle}
\end{figure}

按前面的定义计算收敛系数$C(r)$($0\le r\le 1$),见\autoref{fig:problem_9_conv_speed}。
可以发现,图像行为与\autoref{fig:conv_speed} 十分相像。

\begin{figure}
    \centering
    \includegraphics[width=.8\textwidth]{figures/p9.5_speed_line.pdf}
    \caption{收敛系数$C(r)$}
    \label{fig:problem_9_conv_speed}
\end{figure}

\subsection{自相似行为}
绘制完整的\autoref{fig:problem_9.4} 于\autoref{fig:problem_9.6_full}。
与前面的Logistic map十分相像。
稍取几个分叉点观察自相似行为,见\autoref{fig:problem_9.6_right}。

\begin{figure}
    \centering
    \includegraphics[width=0.8\textwidth]{figures/p9.6_1_0.pdf}
    \caption{序列收敛点$x^*$随$r$的关系}
    \label{fig:problem_9.6_full}
\end{figure}

\begin{figure}
    \centering
    \subfloat[周期2]{
        \label{fig:problem_9.6_right_1}
        \includegraphics[width=0.48\textwidth]{figures/p9.6_1_2.pdf}
    }
    \subfloat[周期4]{
        \label{fig:problem_9.6_right_2}
        \includegraphics[width=0.48\textwidth]{figures/p9.6_1_3.pdf}
    }
    \caption{序列收敛点$x^*$随$r$的关系,左下角取为各分叉点。子图标题中的周期指图片中最左侧的那条分支的周期数}
    \label{fig:problem_9.6_right}
\end{figure}

对各分叉点的$r$进行测量,结果如\autoref{tab:prob_9_r_and_F} 所示。
相邻分叉点的间距$\Delta r$近似以比例4.6缩小。
通过不断放大绘图,得到$r_\infty \approx 0.86557934$。

\begin{table}
    \centering
    \caption{各分叉点处的$r$值}
    \label{tab:prob_9_r_and_F}
    \begin{tabular}{lll}
    \toprule
    \multicolumn{1}{c}{$r$} & \multicolumn{1}{c}{$\Delta r$} & \multicolumn{1}{c}{$F$} \\ \midrule
    0.317 & 0.402 & 3.53 (舍去) \\
    0.719 & 0.1138 & 4.431 \\
    0.8328 & 0.02568 & 4.601 \\
    0.85848 & 0.005582 & 4.691 \\
    0.864062 & 0.001190 & 4.627 \\
    0.865252 & 0.0002572 & 4.72 \\
    0.8655092 & 0.0000545 &  \\
    0.8655637 & \multicolumn{1}{r}{平均:}& 4.6 \\ \bottomrule
    \end{tabular}
\end{table}


\appendix

\chapter{\texttt{Misc}数值库接口}
本人将一些基本的矩阵类型以及相关算法写为\textsf{C++}头文件以方便调用。
本附录将简要介绍可用的接口。
所有的类以及函数均定义于命名空间\texttt{Misc}中,故下文不将\texttt{Misc::}显式写出。

可以通过引入\texttt{"Matrix\_Catalogue.h"}头文件引入所有矩阵类型以及有关运算符重载。
可以通过引入\texttt{"linear\_eq\_direct.h"}与\texttt{"linear\_eq\_iterative.h"}头文件调用矩阵的直接解法与迭代解法。

\section{矩阵}
所有的矩阵类型都继承自虚基类\texttt{template <typename T, size\_t R, size\_t C> class Base\_Matrix},其中模板参数\texttt{T}为数据类型,\texttt{R}, \texttt{C}分别为行数和列数。
注意,\textbf{所有矩阵规模参数都需要编译时指定}。
继承树为
{
\small
\begin{verbatim}
Base_Matrix<T, R, C> 虚基类
|Band_Matrix<T, N, M> 长宽为N,半带宽为M的带状矩阵
||Base_Half_Band_Matrix<T, N, M> 虚基类,长宽为N,半带宽为M
|||Low_Band_Matrix<T, N, M> 长宽为N,半带宽为M的下三角带状矩阵
|||Symm_Band_Matrix<T, N, M> 长宽为N,半带宽为M的对称带状矩阵
|||Up_Band_Matrix<T, N, M> 长宽为N,半带宽为M的上三角带状矩阵
|Base_Tri_Matrix<T, N> 虚基类,长宽为N
||Hermite_Matrix<T, N> 厄密矩阵,长宽为N
||Low_Tri_Matrix<T, N> 下三角矩阵,长宽为N
||Symm_Matrix<T, N> 对称矩阵,长宽为N
||Up_Tri_Matrix<T, N> 上三角矩阵,长宽为N
|Diag_Matrix<T, N> 对角矩阵,长宽为N
|Matrix<T, R, C> 一般矩阵,R行C列
|Sparse_Matrix<T, R, C> 稀疏矩阵;同时继承自std::array<std::map<size_t, T>, R>
\end{verbatim}
}

所有矩阵均提供的接口有
{
\linespread{1.0}
\begin{lstlisting}
Row<T, R, C> row(size_type pos);
const Row<T, R, C> row(size_type pos) const; // 返回某一行
Row<T, R, C> operator[](size_type pos);
const Row<T, R, C> operator[](size_type pos) const; // 同row()
Column<T, R, C> column(size_type pos);
const Column<T, R, C> column(size_type pos) const; // 返回某一列

const Transpose<T, C, R> trans() const; // 返回仅作为右值的转置

T &operator()(size_type row, size_type col);
const T &operator()(size_type row, size_type col); // 返回元素

using p_ta = T (*)[C];
p_ta data();
const p_ta data() const; // 返回内部存储数据的指针
constexpr size_type size() const; // 返回作为一个矩阵的元素个数
size_type data_size() const; // 返回存储的元素个数
size_type data_lines() const; // 返回存储的元素行数
constexpr size_type rows() const; // 返回矩阵行数
constexpr size_type cols() const; // 返回矩阵列数
std::pair<size_type, size_type> shape() const; // 返回矩阵形状
\end{lstlisting}
}

\texttt{row()}和\texttt{column()}方法返回\texttt{Row<T, R, C>}或\texttt{Column<T, R, C>}对象,可作为左值或右值。
可以通过\texttt{[]}下标运算符取或修改原矩阵的元素。

使用标准库类型\texttt{std::array<T, N>}作为矢量。

重载了矩阵间、矩阵与\texttt{Row}或\texttt{Column}或\texttt{array}之间的\texttt{*}运算符,可进行矩阵乘法。

所有矩阵均支持默认初始化,可以提供一个默认值。
如不提供,默认为0。

定义了一些类型间转换,可以把一些特殊的矩阵转化为更一般的矩阵。
比如,\texttt{Diag\_Matrix}可以转换为\texttt{Band\_Matrix}、\texttt{Up\_Tri\_Matrix}等等矩阵。

同时还定义了通过初始化器的列表初始化以便字面指定矩阵。
下面主要介绍一下这一初始化的使用。

\subsection{带状矩阵\texttt{Band\_Matrix}等}
需要沿对角线方向从右上到左下逐行输入,比如
{
\small
\begin{verbatim}
{
    {4, 6, 8, 3, 7},
    {3, 4, 7, 7, 4, 4},
    {2, 5, 5, 5, 2, 1, 7},
    {6, 3, 4, 4, 3, 6},
    {2, 7, 9, 2, 5}
}
\end{verbatim}
}
会得到矩阵\[
    \mqty(
        2&3&4&0&0&0&0\\
        6&5&4&6&0&0&0\\
        2&3&5&7&8&0&0\\
        0&7&4&5&7&3&0\\
        0&0&9&4&2&4&7\\
        0&0&0&2&3&1&4\\
        0&0&0&0&5&6&7
    ).
\]
输入格式错误会引发运行时错误。

对于半带状矩阵以及对称带状矩阵,初始化器的输入顺序为\textbf{从对角线向非零一侧}。

\subsection{三角矩阵\texttt{Base\_Tri\_Matrix}等}
水平从上到下依次输入半侧元素。
比如,
{
\small
\begin{verbatim}
{
    {4},
    {3, 4},
    {2, 1, 7},
    {6, 3, 4, 6},
    {2, 7, 9, 2, 5}
}
\end{verbatim}
}
会得到下三角矩阵\[
    \mqty(
        4&&&&\\
        3&4&&&\\
        2&1&7&&\\
        6&3&4&6&\\
        2&7&9&2&5
    )
\]
或对称矩阵\[
    \mqty(
        4&3&2&6&2\\
        3&4&1&3&7\\
        2&1&7&4&9\\
        6&3&4&6&2\\
        2&7&9&2&5
    ).
\]

但对应的上三角矩阵需要这样输入,
{\small
\begin{verbatim}
{
    {2, 7, 9, 2, 5},
    {6, 3, 4, 6},
    {2, 1, 7},
    {3, 4},
    {4}
}
\end{verbatim}
}
得到\[
    \mqty(
        2&7&9&2&5\\
        0&6&3&4&6\\
        0&0&2&1&7\\
        0&0&0&3&4\\
        0&0&0&0&4
    ).
\]

\subsection{对角矩阵\texttt{Diag\_Matrix}}
直接在一个初始化器列表内输入所有对角元即可。
也可通过提供一对迭代器初始化。

\section{矩阵的直接解法}
\texttt{"linear\_eq\_direct.h"}中主要是对矩阵进行分解的算法,提供\begin{enumerate}
    \item 三角矩阵回代 \texttt{back_sub()}
    \item LU分解 \texttt{lu\_factor()}
    \item LDL分解 \texttt{ldl\_factor()}
    \item Gauss-Jordan法求逆矩阵 \texttt{inv()};对特殊矩阵也会调用用其他算法
    \item 三对角矩阵追赶法 \texttt{tri\_factor()}
    \item Cholesky分解 \texttt{cholesky}
    \item 行列式计算 \texttt{det()}
\end{enumerate}

所有算法(除\texttt{det()})外均提供原处修改的接口与非原处修改接口。
如果仅传入待分解矩阵,那么会进行原处修改;如果也传入了输出矩阵,那么不会修改原矩阵。

\texttt{det()}函数会返回行列式的值。

\section{矩阵的迭代解法}
\texttt{"linear\_eq\_iterative.h"}中主要是对线性方程组进行迭代求解的算法,提供\begin{enumerate}
    \item Jacobi迭代法 \texttt{jacobi()}
    \item Gauss-Seidel迭代法 \texttt{gauss\_seidel()}
    \item 超松弛迭代法 \texttt{suc\_over\_rel()}
    \item 共轭梯度法 \texttt{grad\_des()}
\end{enumerate}

所有算法的接口是基本统一的,参数列表如下\begin{enumerate}
    \item \verb|const Base_Matrix<T, N, N> &in_mat| 系数矩阵
    \item \verb|const std::array<T, N> &in_b| 非齐次项
    \item \verb|std::array<T, N> &out_x| 解向量的初始值;输出的解向量
    \item \verb|T omega| 仅超松弛迭代使用的系数
    \item \verb|bool sparse = false| 是否为稀疏矩阵;如是,那么有优化
    \item \verb|size_t max_times = 1000| 最大迭代次数
    \item \verb|double rel_epsilon = 1e-15| 判停标准:残差与初始残差的1-范数之比小于此值时即停止
\end{enumerate}


\end{document}
