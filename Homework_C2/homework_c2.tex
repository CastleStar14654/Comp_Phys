\documentclass[a4paper,unicode]{report}
\usepackage{amsmath}
\usepackage{amssymb}
\usepackage{textgreek}
\usepackage{esint}
\usepackage{subfig}
\usepackage{rotating}
\usepackage{booktabs}
\usepackage{longtable}
\usepackage{paralist}
\usepackage{titlesec}
\usepackage{graphicx}
\usepackage{physics}
\usepackage{ucs}
\usepackage{listings}
\usepackage{multirow}
\usepackage{tablefootnote}
\usepackage{indentfirst}
\usepackage{tikz}
\usepackage{warpcol}
\usepackage[pdftitle=计算物理第三次大作业第二部分,pdfauthor=***,bookmarksnumbered=true]{hyperref}
\usepackage{natbib}
\usepackage{xcolor}
\usepackage{xeCJK}
    \setCJKmainfont[BoldFont={Noto Serif CJK SC Bold},ItalicFont={FangSong}]{Noto Serif CJK SC}
    \setCJKsansfont[BoldFont={Noto Sans CJK SC Bold},ItalicFont={KaiTi}]{Noto Sans CJK SC}
    \setCJKmonofont[BoldFont={Noto Sans Mono CJK SC Bold}]{Noto Sans Mono CJK SC}

\title{计算物理第三次大作业第二部分}
\author{物理学院\quad ***\quad 1800011***}

\DeclareUnicodeCharacter{"00B0}{\textdegree}
\DeclareUnicodeCharacter{"00B5}{\textmu}

\renewcommand{\today}{\number\year 年\number\month 月\number\day 日}
\renewcommand{\refname}{参考文献}
\renewcommand{\abstractname}{摘要}
\renewcommand{\contentsname}{目录}
\renewcommand{\figurename}{图}
\renewcommand{\tablename}{表}
\renewcommand{\appendixname}{附录}
\renewcommand{\chaptername}{第}
\newcommand{\chapterendname}{章}

\renewcommand{\equationautorefname}{式}
\renewcommand{\figureautorefname}{图}
\newcommand{\subfigureautorefname}{子图}
\renewcommand{\tableautorefname}{表}
\renewcommand{\subsectionautorefname}{小节}
\newcommand{\mythmautorefname}{定理}
\renewcommand{\sectionautorefname}{\S}

\newtheorem{mythm}{定理}

\lstset{
    basicstyle=\scriptsize\ttfamily\color{black}, % print whole listing small
    keywordstyle=\color{teal}\bfseries,
    identifierstyle=, % nothing happens
    commentstyle=\color{gray}\ttfamily, % white comments
    stringstyle=\color{violet}, % typewriter type for strings
    showstringspaces=true,
    numbers=left,
    numberstyle=\tiny\color{brown},
    stepnumber=2,
    numbersep=5pt,
    firstnumber=auto,
    frame=lines,
    language=[11]C++,
    rangeprefix=/*,
    rangesuffix=*/,
    includerangemarker=false
}

% \pgfsetxvec{\pgfpoint{0.02cm}{0}}
% \pgfsetyvec{\pgfpoint{0}{0.1em}}

% \usetikzlibrary{datavisualization,plotmarks,datavisualization.formats.functions}
% \usetikzlibrary{math,fpu,datavisualization}
% \graphicspath{{figure/}}
\linespread{1.3}

\titleformat{\chapter}[display]{\huge\bfseries}{\chaptertitlename\ \thechapter\ \chapterendname}{20pt}{\Huge}

\begin{document}

\maketitle
\tableofcontents

\begin{center}
    \textbf{编译与运行环境说明}
\end{center}

本地操作系统版本为\textsf{Windows 10 x64 1909 (18363.836)}。

第二次大作业第一题主要使用\textsf{C++}编写,编译器为Windows下的\texttt{g++ 10.1.0} (Rev3, Built by MSYS2 project),标准为\textsf{C++17}。
除标准库外未使用其他第三方库。
最后上交的程序的编译命令为\begin{verbatim}
    g++ -O3 -static -std=c++17 -o <可执行文件名> <源文件名>
\end{verbatim}

对于绘图部分,使用了必要的\textsf{Python}的第三方库\textsf{NumPy}和\texttt{matplotlib}进行处理。

\setcounter{chapter}{1}
\chapter{玻尔兹曼方程}
玻尔兹曼方程是描述分布函数$ f = f(\vb{r},\vb{p},t)$在相空间的演化的偏微分方程,一般形式为
\begin{equation}\label{eq:boltzmann}
    \pdv{f}{t} + \frac{\vb{p}}{m}\cdot\grad f + \vb{F}\cdot\pdv{f}{\vb{p}} = \qty(\pdv{f}{t})_\text{coll},
\end{equation}
其中等式右侧为碰撞项。
在弛豫时间近似下,有
\begin{equation}
    \qty(\pdv{f}{t})_\text{coll} = -\nu (f -f_0),
\end{equation}
其中$f_0$为平衡分布(即麦克斯韦--玻尔兹曼分布),而$\nu$为碰撞频率。

本题考虑这样一个热力学系统的演化:
有两个体积为 V 的方盒,竖直叠放,中间有一挡板。
盒子内各有$N$个种类相同的粒子,从下到上分别与温度为$T_1$和$T_2$的大热源接触,其中的粒子已经达到了稳定的玻尔兹曼分布。
之后我们去除挡板,再将盒子与温度为$T_3$的大热源相接触。

为简化问题,忽略各物理量的单位。
考虑$T_1 = 10$,$T_2 = 20$,$T_3 = 30$,$N = 1$。
仅考虑$z$方向,每个盒子$z$方向高度固定为 1。
同时令 $k = 1, m = 1$,碰撞频率$\nu = 100$,重力加速度取$g = 10$。
通过认为粒子在到达空间边界时发生弹性碰撞来使粒子数守恒。

此外,玻尔兹曼基于玻尔兹曼方程也提出了用于判断体系演化方向的 H 定理。
状态函数
\begin{equation}
    H(t) = \int f(\vb{r},\vb{p},t)\ln f(\vb{r},\vb{p},t)\dd\vb{r}\dd\vb{p}
\end{equation}
将向极小值演化。
接下来将会利用状态函数$H(t)$做一些说明。

\section{不考虑重力时的初始态}
简化起见,先不考虑重力。
则空间上粒子将在两个盒子里均匀分布,而动量则遵循麦克斯韦分布。
即
\begin{equation}
    f(z, p, 0) = \begin{cases}
        \frac{1}{\sqrt{2\pi T_1}}\exp(-\frac{p^2}{2T_1}), & z < 0,\\
        \frac{1}{\sqrt{2\pi T_2}}\exp(-\frac{p^2}{2T_2}), & z \ge 0,\\
    \end{cases}
\end{equation}
其中系数已做了归一化。

使用如下函数构造这样的平衡分布于数组\verb|data|。
{
    \linespread{1.0}
    \lstinputlisting[linerange=beg:equi_dec-end:equi_dec]{2_Boltzmann_eq.cpp}
    \lstinputlisting[linerange=beg:equi-end:equi]{2_Boltzmann_eq.cpp}
}
其中参数\verb|T|为温度,\verb|g|为预留的引入玻尔兹曼分布的接口,\verb|from| \verb|to|为仅处理一个盒子时指定$z$方向上的起始坐标使用。
函数的思路即为先计算系数,再对各个数据点赋值。

绘制得到$z\in [-1, 1], p\in [-20, 20]$中的分布函数于\autoref{fig:1}。
可以看到,高处 ($T_2$) 的盒子中的动量弥散较大。

\begin{figure}
    \centering
    \includegraphics[width=0.9\textwidth]{figures/1.pdf}
    \caption{初始无重力时的分布函数}
    \label{fig:1}
\end{figure}

\section{无重力的演化}
下面考虑离散化求解方程 \eqref{eq:boltzmann},使用迎风格式求解。

首先是网格离散化。
取$p$绝对值最大为20,在$p, z$上各均匀分为200段,取各小区间的中心为数据点。
注意到,对时间步长$\tau$,有要求
\begin{equation}
    \tau \le \frac{h_z}{p}, \quad \tau \le \frac{h_p}{F},
\end{equation}
可得到,前者在$p=20$时对$\tau$加了最强的限制,$\tau\le 0.0005$。
保险起见,取$\tau = 0.0005$。
其他一些预定义参数如下
{
    \linespread{1.0}
    \lstinputlisting[linerange=beg:grid-end:grid]{2_Boltzmann_eq.cpp}
}
其中\verb|p_s|, \verb|z_s|为各格点的$p, z$坐标,初始化如下
{
    \linespread{1.0}
    \lstinputlisting[linerange=beg:init-end:init]{2_Boltzmann_eq.cpp}
}

\subsection{迎风格式}
\eqref{eq:boltzmann} 即
\begin{equation}
    \pdv{f}{t} + p\pdv{z} f - g\pdv{p} f = -\nu (f -f_0), \tag{\ref{eq:boltzmann}b}
\end{equation}
按迎风格式离散化可得
\begin{equation}\label{eq:upwind}
    \frac{f_{ij}^{(m+1)} - f_{ij}^{(m)}}{\tau}
    + p_{ij}^{(m)} \pdv{z} f
    - g \frac{f_{i,j+1}^{(m)} - f_{ij}^{(m)}}{h_p}
    = -\nu (f_{ij}^{(m)} - f_{ij}^{(\text{eq})}),
\end{equation}
其中
\begin{equation}
    \pdv{z} f = \begin{cases}
        \frac{f_{i,j}^{(m)} - f_{i-1,j}^{(m)}}{h_z}, & p_{ij}^{(m)} \ge 0,\\
        \frac{f_{i+1,j}^{(m)} - f_{i,j}^{(m)}}{h_z}, & p_{ij}^{(m)} \le 0.
    \end{cases}
\end{equation}

根据式 \eqref{eq:upwind},可得到单步迭代的函数
{
    \linespread{1.0}
    \lstinputlisting[linerange=beg:step_dec-end:step_dec]{2_Boltzmann_eq.cpp}
    \lstinputlisting[linerange=beg:step-end:step]{2_Boltzmann_eq.cpp}
}

其中有涉及到关于边界条件的处理,处理方式如下。
\begin{enumerate}
    \item 对于$p$方向上越界的情况,直接截断。
    \item 对于$x$方向上越界的情况,假设其访问的越界格点为$x$值相同,而$p$值相反的格点。根据迎风格式的特点,这一格点恰对应将要被弹性反弹回来的粒子。
\end{enumerate}

\subsection{分布函数的演化}
使用如下函数生成稳定分布\verb|*p_equi_array|,以及初始的分布\verb|*current|。
使用\verb|*next_frame|临时存储下一帧的内容。
每10帧(0.0040时间单位)输出一次数据,输出前15次,迭代到200帧(0.0800时间单位)。
{
    \linespread{1.0}
    \lstinputlisting[linerange=end:init-end:problem_2]{2_Boltzmann_eq.cpp}
}
其中\verb|output|函数为自定义的二进制输出。
此外,以上代码还计算了$H(t)$(见后)。

计算得到的分布函数见\autoref{fig:f_no_g}。
由于到0.0280时间单位时,肉眼已难以看出与平衡态的区别,故仅绘制前8张。
注意每一张的色标不同。

\begin{figure}
    \centering
    \subfloat{
        \includegraphics[width=0.48\textwidth]{figures/2_0.0000.pdf}
    }
    \subfloat{
        \includegraphics[width=0.48\textwidth]{figures/2_0.0040.pdf}
    }\\
    \subfloat{
        \includegraphics[width=0.48\textwidth]{figures/2_0.0080.pdf}
    }
    \subfloat{
        \includegraphics[width=0.48\textwidth]{figures/2_0.0120.pdf}
    }\\
    \subfloat{
        \includegraphics[width=0.48\textwidth]{figures/2_0.0160.pdf}
    }
    \subfloat{
        \includegraphics[width=0.48\textwidth]{figures/2_0.0200.pdf}
    }\\
    \subfloat{
        \includegraphics[width=0.48\textwidth]{figures/2_0.0240.pdf}
    }
    \subfloat{
        \includegraphics[width=0.48\textwidth]{figures/2_0.0280.pdf}
    }
    \caption{无重力时,分布函数$f$的演化}
    \label{fig:f_no_g}
\end{figure}

可以看到,图上出现了“顺时针”的混合,最后变为均匀的、$p$上的高斯分布。

将$z$或$p$方向上进行求和,得到不同时间的$p$上或$z$上的分布函数于\autoref{fig:f_zp_no_g}。
可以看到,由于升温,$p$上的高斯分布峰高降低、展宽增大。
$x$上的分布则为一个向两侧传播、最终重新平缓的震荡。

\begin{figure}
    \centering
    \subfloat[分布函数与$z$的关系]{
        \includegraphics[width=0.85\textwidth]{figures/2_f_z.pdf}
    }\\
    \subfloat[分布函数与$p$的关系]{
        \includegraphics[width=0.85\textwidth]{figures/2_f_p.pdf}
    }
    \caption{无重力时,分布函数$f$的演化}
    \label{fig:f_zp_no_g}
\end{figure}

\subsection{\texorpdfstring{$H(t)$}{H(t)}的演化}
前面的代码中调用了计算$H(t)$的函数\verb|H_status()|,定义如下
{
    \linespread{1.0}
    \lstinputlisting[linerange=beg:H-end:H]{2_Boltzmann_eq.cpp}
}

得到的本小题中$H(t)$的演化如\autoref{fig:H_no_g} 所示。
$H(t)$最后大致稳定于$-6.23$。

\begin{figure}
    \centering
    \includegraphics[width=0.85\textwidth]{figures/2_H.pdf}
    \caption{无重力时,$H$的演化}
    \label{fig:H_no_g}
\end{figure}

\section{有重力的演化}
有重力时,初始的分布函数为
\begin{equation}
    f(z, p, 0) = \begin{cases}
        \frac{1}{\sqrt{2\pi T_1}}\frac{g/2T}{\exp(-g/2T)\sinh(g/2T)} \exp(-\frac{p^2}{2T_1}-\frac{gz}{T_1}), & z < 0,\\
        \frac{1}{\sqrt{2\pi T_2}}\frac{g/2T}{\exp(g/2T)\sinh(g/2T)} \exp(-\frac{p^2}{2T_2}-\frac{gz}{T_2}), & z \ge 0,\\
    \end{cases}
\end{equation}
最终稳态为
\begin{equation}
    f^{(\text{eq})}(z, p, 0) =
        \frac{1}{\sqrt{2\pi T_3}}
        \frac{g/T}{\sinh(g/T)}
        \exp(-\frac{p^2}{2T_3}-\frac{gz}{T_3}).
\end{equation}
只需向前面定义的接口中传入重力加速度$g$,即可容易求解。

得到的分布函数随时间的变化如\autoref{fig:f_g} 所示。
可以发现,初始时($t=0$),粒子堆积在两个盒子的底端,同时温度较低的盒子中粒子在动量空间上更为密集。

\begin{figure}
    \centering
    \subfloat{
        \includegraphics[width=0.48\textwidth]{figures/3_0.0000.pdf}
    }
    \subfloat{
        \includegraphics[width=0.48\textwidth]{figures/3_0.0040.pdf}
    }\\
    \subfloat{
        \includegraphics[width=0.48\textwidth]{figures/3_0.0080.pdf}
    }
    \subfloat{
        \includegraphics[width=0.48\textwidth]{figures/3_0.0120.pdf}
    }\\
    \subfloat{
        \includegraphics[width=0.48\textwidth]{figures/3_0.0160.pdf}
    }
    \subfloat{
        \includegraphics[width=0.48\textwidth]{figures/3_0.0200.pdf}
    }\\
    \subfloat{
        \includegraphics[width=0.48\textwidth]{figures/3_0.0240.pdf}
    }
    \subfloat{
        \includegraphics[width=0.48\textwidth]{figures/3_0.0280.pdf}
    }
    \caption{有重力时,分布函数$f$的演化}
    \label{fig:f_g}
\end{figure}

随着时间的推移,同样是经过相空间上一个“顺时针”的演化,粒子到达了堆积到大盒子底部的分布。

更清晰的$p$上或$z$上的分布函数见\autoref{fig:f_zp_g}。
可以看到,由于升温,$p$上的高斯分布峰高降低、展宽增大。
$x$上的分布则由两个分离的指数分布演化为同一个指数分布。

\begin{figure}
    \centering
    \subfloat[分布函数与$z$的关系]{
        \includegraphics[width=0.85\textwidth]{figures/3_f_z.pdf}
    }\\
    \subfloat[分布函数与$p$的关系]{
        \includegraphics[width=0.85\textwidth]{figures/3_f_p.pdf}
    }
    \caption{有重力时,分布函数$f$的演化}
    \label{fig:f_zp_g}
\end{figure}

$H(t)$的演化见\autoref{fig:H_g}。
$H(t)$最后大致稳定于$-6.20$。

\begin{figure}
    \centering
    \includegraphics[width=0.85\textwidth]{figures/3_H.pdf}
    \caption{有重力时,$H$的演化}
    \label{fig:H_g}
\end{figure}

\end{document}
