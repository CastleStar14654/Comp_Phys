\documentclass[a4paper,unicode]{report}
\usepackage{amsmath}
\usepackage{amssymb}
\usepackage{textgreek}
\usepackage{esint}
\usepackage{subfig}
\usepackage{rotating}
\usepackage{booktabs}
\usepackage{longtable}
\usepackage{paralist}
\usepackage{titlesec}
\usepackage{graphicx}
\usepackage{physics}
\usepackage{ucs}
\usepackage{listings}
\usepackage{multirow}
\usepackage{tablefootnote}
\usepackage{indentfirst}
\usepackage{tikz}
\usepackage{warpcol}
\usepackage[pdftitle=计算物理第三次大作业第一部分,pdfauthor=***,bookmarksnumbered=true]{hyperref}
\usepackage{natbib}
\usepackage{xcolor}
\usepackage{xeCJK}
    \setCJKmainfont[BoldFont={Noto Serif CJK SC Bold},ItalicFont={FangSong}]{Noto Serif CJK SC}
    \setCJKsansfont[BoldFont={Noto Sans CJK SC Bold},ItalicFont={KaiTi}]{Noto Sans CJK SC}
    \setCJKmonofont[BoldFont={Noto Sans Mono CJK SC Bold}]{Noto Sans Mono CJK SC}

\title{计算物理第二次大作业第二部分}
\author{物理学院\quad ***\quad 1800011***}

\DeclareUnicodeCharacter{"00B0}{\textdegree}
\DeclareUnicodeCharacter{"00B5}{\textmu}

\renewcommand{\today}{\number\year 年\number\month 月\number\day 日}
\renewcommand{\refname}{参考文献}
\renewcommand{\abstractname}{摘要}
\renewcommand{\contentsname}{目录}
\renewcommand{\figurename}{图}
\renewcommand{\tablename}{表}
\renewcommand{\appendixname}{附录}
\renewcommand{\chaptername}{第}
\newcommand{\chapterendname}{章}

\renewcommand{\equationautorefname}{式}
\renewcommand{\figureautorefname}{图}
\newcommand{\subfigureautorefname}{子图}
\renewcommand{\tableautorefname}{表}
\renewcommand{\subsectionautorefname}{小节}
\newcommand{\mythmautorefname}{定理}
\renewcommand{\sectionautorefname}{\S}

\newtheorem{mythm}{定理}

\lstset{
    basicstyle=\scriptsize\ttfamily\color{black}, % print whole listing small
    keywordstyle=\color{teal}\bfseries,
    identifierstyle=, % nothing happens
    commentstyle=\color{gray}\ttfamily, % white comments
    stringstyle=\color{violet}, % typewriter type for strings
    showstringspaces=true,
    numbers=left,
    numberstyle=\tiny\color{brown},
    stepnumber=2,
    numbersep=5pt,
    firstnumber=auto,
    frame=lines,
    language=[11]C++,
    rangeprefix=/*,
    rangesuffix=*/,
    includerangemarker=false
}

% \pgfsetxvec{\pgfpoint{0.02cm}{0}}
% \pgfsetyvec{\pgfpoint{0}{0.1em}}

% \usetikzlibrary{datavisualization,plotmarks,datavisualization.formats.functions}
% \usetikzlibrary{math,fpu,datavisualization}
% \graphicspath{{figure/}}
\linespread{1.3}

\titleformat{\chapter}[display]{\huge\bfseries}{\chaptertitlename\ \thechapter\ \chapterendname}{20pt}{\Huge}

\begin{document}

\maketitle
\tableofcontents

\begin{center}
    \textbf{编译与运行环境说明}
\end{center}

本地操作系统版本为\textsf{Windows 10 x64 1909 (18363.836)}。

第二次大作业第一题主要使用\textsf{C++}编写,编译器为Windows下的\texttt{g++ 10.1.0} (Rev2, Built by MSYS2 project),标准为\textsf{C++17}。
除标准库外未使用其他第三方库。
最后上交的程序的编译命令为\begin{verbatim}
    g++ -O3 -static -std=c++17 -o <可执行文件名> <源文件名>
\end{verbatim}

对于绘图部分,使用了必要的\textsf{Python}的第三方库\textsf{NumPy}和\texttt{matplotlib}进行处理。

\chapter{绝热不变量}

对于哈密顿系统$H(q,p)$,若其在相空间中的等能相曲线闭合,可定义绝热不变量
\begin{equation}
    J\equiv \frac{1}{2\pi}\oint p\dd q.
\end{equation}

若哈密顿量对时间的依赖是通过$H$对缓变参量$\lambda(t)$的依赖实现的,即$H(q,p;\lambda(t))$,则此时绝热不变量近似与$\lambda$无关,
\begin{equation}
    \lim_{\dv{t}\lambda\to 0}\dv{J}{\lambda} = 0.
\end{equation}

本题考查\autoref{fig:schematic} 所示的力学系统的绝热不变量。
其哈密顿量(不含时)可写为
\begin{equation}
    H(y, p_y) = \frac{p_y^2}{2m} + \frac{1}{2}k\qty(\sqrt{x^2+y^2} - l)^2 + mgy,
\end{equation}
正则运动方程为
\begin{subequations}\label{eq:canonical_eq}
\begin{align}
    \dot{y} &{}= \frac{p_y}{m},\\
    \dot{p_y} &{}= -mg-ky\qty(1-\frac{l^2}{\sqrt{x^2+y^2}}).
\end{align}
\end{subequations}

本题的程序为\verb|1_adiabatic_invariant.exe|。
调用时,需要用\verb|-<小题号>|的格式传入需要求解的小题。
启动文件夹内可能需要存在\verb|output/|文件夹。
\verb|1_graph.py|可根据结果得到图像。

\begin{figure}
    \centering
    \includegraphics[width=0.5\textwidth]{figures/schematic.pdf}
    \caption{力学模型示意图}
    \label{fig:schematic}
\end{figure}

\section{系统平衡位置}\label{sec:equiburium}

系统平衡位置满足
\begin{equation}\label{eq:equiburium_criteria}
    \dot{p_y} = 0 = -mg-ky\qty(1-\frac{l^2}{\sqrt{x^2+y^2}}).
\end{equation}
做如下无量纲化操作
\begin{equation}
    \xi := \frac{x}{l},\quad
    \eta := \frac{y}{l},\quad
    \bar{\eta} := \frac{\eta}{\qty|\xi|},\quad
    A := \frac{mg}{kl}.
\end{equation}
则式 \eqref{eq:equiburium_criteria} 可写为
\begin{equation}
    \qty|\xi| + \frac{A}{\bar{\eta}} = \frac{1}{\sqrt{\bar{\eta}^2 + 1}}.
\end{equation}
平衡位置$\bar{\eta}$为上式的解,参量为$A$与$|\xi|$。

注意到,对于$A<0$的情形,我们只需将$\bar{\eta}$也反号,那么就又和$A>0$一样了。
故一下以$A\ge 0$讨论。

考虑平衡位置的个数。显然,在$\bar{\eta}<0$处一定有一个平衡位置。
而在$\bar{\eta} > 0$则可能出现0至2个平衡位置。
详见\autoref{fig:equiburium}。

\begin{figure}
    \centering
    \subfloat[$\xi=0.5,A=0.4$]{
        \includegraphics[width=0.48\textwidth]{figures/0.5_0.4.pdf}
    }
    \subfloat[$\xi=0.8,A=0.2$]{
        \includegraphics[width=0.48\textwidth]{figures/0.8_0.2.pdf}
    }\\
    \subfloat[$\xi=0.5,A=0.2$]{
        \includegraphics[width=0.7\textwidth]{figures/0.5_0.2.pdf}
    }
    \caption{平衡位置个数示意。蓝色线为$\qty|\xi| + \frac{A}{\bar{\eta}}$,橙色线为$\frac{1}{\sqrt{\bar{\eta}^2 + 1}}$。}
    \label{fig:equiburium}
\end{figure}

可以发现,$A$或$\qty|\xi|$过大均会导致$\eta>0$处没有平衡位置。
下面求$\eta>0$处恰没有平衡位置的临界条件。
记$f(\eta) = \qty|\xi| + \frac{A}{\bar{\eta}}$,$g(\eta) = \frac{1}{\sqrt{\bar{\eta}^2 + 1}}$,分析\autoref{fig:equiburium},有$\eta>0$的相切条件
\begin{equation}
    f(\eta) = g(\eta),\quad f'(\eta) = g'(\eta),\quad f''(\eta) > g''(\eta).
\end{equation}
即
\begin{equation}
    \qty|\xi| + \frac{A}{\bar{\eta}} = \frac{1}{\sqrt{\bar{\eta}^2 + 1}},\quad
    -\frac{A}{\bar{\eta}^2} = -\frac{\bar{\eta}}{\qty(\bar{\eta}^2 + 1)^\frac{3}{2}},
\end{equation}
解得
\begin{equation}\label{eq:A_xi}
    A = \frac{\bar{\eta}^3}{\qty(\bar{\eta}^2 + 1)^\frac{3}{2}},\quad
    \qty|\xi| = \frac{1}{\qty(\bar{\eta}^2 + 1)^\frac{3}{2}},
\end{equation}
满足条件$\frac{2A}{\bar{\eta}^3} > \frac{2\bar{\eta}^2 - 1}{\qty(\bar{\eta}^2 + 1)^\frac{5}{2}}$。

我们可以把式 \eqref{eq:A_xi} 进一步写成
\begin{equation}
    \xi^\frac{2}{3} + A^\frac{2}{3}
    = \qty(\frac{x}{l})^\frac{2}{3} + \qty(\frac{mg}{kl})^\frac{2}{3} = 1, \quad A > 0,
\end{equation}
考虑上$A<0$的情况,绘图如\autoref{fig:x_g} 所示。

\begin{figure}
    \centering
    \includegraphics[width=0.8\textwidth]{figures/x_g.pdf}
    \caption{系统仅存在一个平衡位置与存在多个平衡位置的分界线。$A=\frac{mg}{kl}$的值域为$[0,\infty]$。分界线内部,系统有三个平衡位置;分界线上有两个;外面一个}
    \label{fig:x_g}
\end{figure}

对于$A=g=0$的情况,则是$\qty|\xi|\ge 1$时一个平衡位置,$\qty|\xi|<1$时有两个。

\section{\texorpdfstring{$x$}{x}缓变}\label{sec:slow_x}

以速度 $v$ 缓慢的左移直杆,使$x=2l-vt$。
$g=0, y(0) = 0.1l, p_y(0)=0$。
为方便,对式 \eqref{eq:canonical_eq} 进行无量纲化,取
\begin{equation}
    \eta = \frac{y}{l},\quad
    \xi = \frac{x}{l},\quad
    \mathcal{P} = \frac{p_y}{\sqrt{mk}l},\quad
    \tau = \sqrt{\frac{k}{m}}t.
\end{equation}
得到
\begin{subequations}
    \label{eq:dimensionless}
    \begin{align}
        \dv{\tau} \eta &{}= \mathcal{P},\\
        \dv{\tau} \mathcal{P} &{}= -\frac{mg}{kl} - \eta\qty(1-\frac{1}{\sqrt{\xi^2+\eta^2}}).
    \end{align}
\end{subequations}

为了获得相轨,计算绝热不变量,需要先求解初值微分方程 \eqref{eq:dimensionless},然后将$\mathcal{P}\dv{\tau}\eta=\mathcal{P}^2$对$\tau$插值,从而积分获得绝热不变量$\frac{J}{\sqrt{mk}l^2} = \oint \mathcal{P}\dv{\tau}\eta \dd{\tau}$。

\subsection{隐式RK法求解微分方程初值问题}
IVP问题的求解代码在头文件\verb|misc/ivp.h|中。

首先定义了IVP问题求解器的基类\verb|_IVP_solver<N>|,求解问题
\begin{equation}
    \dv{t}\vb{y} =  \vb{f}(t, \vb{y}).
\end{equation}
模板参数\verb|N|为一阶微分方程组的方程个数。
内部存储函数\verb|func|,依次求解过的$t$ \verb|ts|、$\vb{y}$ \verb|ys|、上一次迭代得到的$\vb{y}$ \verb|y|。
提供构造函数、供子类实现的单步迭代函数\verb|step_inc()|(接收当前$t$,步长$h$,$\vb{y}$,返回$\vb{y}$的增量)、单步迭代函数(接收步长$h$和$\vb{y}$的增量,更新数据),以及结束迭代,转发数值解的接口\verb|finish()|。
{
    \linespread{1.0}
    \lstinputlisting[linerange=beg:ivp_solver-end:ivp_solver]{../misc/ivp.h}
}

然后从上派生了使用RADAU 5隐式迭代求解的\verb|_Radau_5<N>|。
其内部先使用向前欧拉法获得各个划分点上的估计值,然后迭代2轮。
其精度相比欧拉法提升很多,运算仍很快速。
{
    \linespread{1.0}
    \lstinputlisting[linerange=beg:radau-end:radau]{../misc/ivp.h}
}

最后定义了包装函数\verb|ivp_radau()|。
其提供了变步长参数选项\verb|auto_h|,但本次未使用。
{
    \linespread{1.0}
    \lstinputlisting[linerange=beg:radau_func-end:radau_func]{../misc/ivp.h}
}

\subsection{三次厄密插值}
三次厄密插值接收自变量、因变量与一阶导数,进行分段三次插值,在已知导数值时是比较适合的插值方法。
自行编写的三次厄密插值见头文件\verb|misc/interpolate/interpolate_spline.h|中的类\verb|Cubic_Hermite<T>|。
由于代码较长,但核心仅为构造时存储点并算出各个区间上的多项式系数、求值时判断$x$位于哪个区间并调用相应的多项式,故不将代码列出。

\verb|Cubic_Hermite<T>|构造时可选择接收迭代器/容器/\verb|initializer_list|作为提供数据的方式。

其中用到了多项式类\verb|Polynomial|。
其派生自标准库\verb|vector|,提供求值、乘除运算、求导积分等接口。

\subsection{方程求根}
计算绝热不变量时,需要判断系统何时完成了一个周期的运动,因此需要数值求根。

自行编写的数值求根工具位于头文件\verb|misc/rootfind.h|中,有二分法\verb|bisection()|,Dekker--Brent方法\verb|dekker_brent()|以及迭代法\verb|steffenson|。
本题使用\verb|dekker_brent()|。
此函数接收待求根函数\verb|func|,求根区间左右界\verb|a|,\verb|b|,以及其他可选参数。

该\verb|dekker_brent()|改编自\emph{Numerical Recipes}上的版本\footnote{\textsf{C++}版本,小节9.3,pp. 454--456},对其运算顺序做了一定优化。
由于代码较长,故不列出。

\subsection{数值积分}
计算绝热不变量,需要对求得的离散数据点插值后进行积分。
数值积分工具位于头文件\verb|misc/integral.h|中。
头文件内定义了蒙卡积分\verb|monte_carlo_integrate()|、Rumberg积分\verb|integrate()|及其可处理奇点版本、可处理两端为奇点(包括无穷)的双对数变换积分\verb|integrate_DE()|。
本题无处理奇点的需求,故使用Rumberg积分。

代码较长,故不列出。
该函数在进行判断积分区间合法性,并变换到$a$小$b$大后,从128划分开始求积分。
使用标准库容器红黑树\verb|map|保存已求过值的函数值,避免重复计算。

之后不断二分划分长度,直到误差小于给定要求。
\subsection{求解相轨}
为方便,定义了类\verb|HamiltonianEq|。
构造实例时,可指定$g, x$与$t$的函数关系。
该类的实例是可调用的,传入$t, \{y, p\}$,会返回数组$\{\dot{y}, \dot{p}\}$。
{
    \linespread{1.0}
    \lstinputlisting[linerange=beg:hamiltonian-end:hamiltonian]{1_adiabatic_invariant.cpp}
}
因此,它的实例可以作为\verb|ivp_radau|的参数。

求解相轨的代码如下
{
    \linespread{1.0}
    \lstinputlisting[linerange=beg:prob_2-end:solve_ivp]{1_adiabatic_invariant.cpp}
}
即为简单地求解一个IVP问题。
本题中,由于$g\equiv 0$,故\verb|func_g()|简单地定义为返回0。
\verb|func_x()|则定义为与$v$有关。
取时间步长为0.1,让$x$从2变为0。

数据输出部分如下,输出为二进制文件:
{
    \linespread{1.0}
    \lstinputlisting[linerange=beg:2_output-beg:prob_3]{1_adiabatic_invariant.cpp}
}
然后在\textsf{Python}中使用\textsf{NumPy}读取、用\textsf{matplotlib}绘图于\autoref{fig:prob2_yp}。

\begin{figure}
    \centering
    \subfloat[$\sqrt{\frac{m}{k}}\frac{v}{l}=\frac{1}{4}$]{
        \includegraphics[width=0.48\textwidth]{figures/adiabatic_1.2_yp4.pdf}
    }
    \subfloat[$\sqrt{\frac{m}{k}}\frac{v}{l}=\frac{1}{16}$]{
        \includegraphics[width=0.48\textwidth]{figures/adiabatic_1.2_yp16.pdf}
    }\\
    \subfloat[$\sqrt{\frac{m}{k}}\frac{v}{l}=\frac{1}{64}$]{
        \includegraphics[width=0.48\textwidth]{figures/adiabatic_1.2_yp64.pdf}
    }
    \subfloat[$\sqrt{\frac{m}{k}}\frac{v}{l}=\frac{1}{256}$]{
        \includegraphics[width=0.48\textwidth]{figures/adiabatic_1.2_yp256.pdf}
    }
    \caption{$x=2-vt$时的系统相轨。时间跨度为$\qty[0, \frac{2}{v}]$}
    \label{fig:prob2_yp}
\end{figure}

对于$g=0$,由 \autoref{sec:equiburium} 的讨论,在$x<l$时,系统平衡位置会分裂为两个,关于$y$的原点对称。
可以发现,\autoref{fig:prob2_yp} 四张图中均可看到这一平衡位置的转变行为。
但是,对于$x$变化得比较激烈的,相轨也出现了大规模变动,甚至不存在周期性运动。
而对于$x$变化得十分缓慢的,相轨则是除了在最初的位置作周期性运动外,还能转化到新的平衡位置作周期运动。

可以猜测,对前者甚至无法计算出良定义的绝热不变量;而对后者,则可对两种周期运动都算出不变量。

\subsection{求解绝热不变量}
求解绝热不变量有两个部分:
\begin{enumerate}
    \item 获得连续的被积函数$\mathcal{P}\dv{\tau}\eta=\mathcal{P}^2$;
    \item 获得每个运动周期的起始时刻。
\end{enumerate}
对于前者,我们使用三次厄密插值;对于后者,我们以$\mathcal{P}=0$为判据,在使用插值获得连续的$\mathcal{P}(t)$后,进行求根操作获得$J$积分的上下限。

\subsubsection{插值\texorpdfstring{$\mathcal{P}\dv{\tau}\eta$}{P d/dτ η}}
由哈密顿正则方程,$\mathcal{P}\dv{\tau}\eta = \mathcal{P}^2$,
\begin{equation}
    \dv{\tau}\mathcal{P}\dv{\tau}\eta = 2\mathcal{P}\dv{\tau}\mathcal{P}.
\end{equation}

计算插值的代码如下。
\verb|p_dots|为$\dv{\tau}\mathcal{P}$, \verb|p_y_dots|为$\mathcal{P}\dv{\tau}\eta$, \verb|p_y_dot__dots|为$\dv{\tau}\mathcal{P}\dv{\tau}\eta$。
将这些变量列表都计算出来,然后插值。
{
    \linespread{1.0}
    \lstinputlisting[linerange=end:solve_ivp-end:spline_p_y_dot]{1_adiabatic_invariant.cpp}
}

\subsubsection{获得周期运动的起始时刻}

起始时刻通过求解所有$\mathcal{P}=0$的点获得。
注意到,这样的相邻两个零点仅有半个周期。

首先插值获得连续的$\mathcal{P}(t)$函数,然后先根据离散的$t, \mathcal{P}$关系获得零点的估计值,存入\verb|p_t_zeroes|。
最后,利用求根函数获得精确零点。
代码如下
{
    \linespread{1.0}
    \lstinputlisting[linerange=end:spline_p_y_dot-end:find_zero]{1_adiabatic_invariant.cpp}
}

\subsubsection{获得绝热不变量}

整体想法是,对于前面求得的零点$\tau_{i-1}, \tau_i, \tau_{i+1}$,将$x(\tau_i)$与$\int_{\tau_{i-1}}^{\tau_{i+1}}\mathcal{P}\dv{\tau}\eta \dd{\tau}$存为一对值。
代码如下
{
    \linespread{1.0}
    \lstinputlisting[linerange=end:find_zero-beg:2_output]{1_adiabatic_invariant.cpp}
}

将获得的$x-J$绘图于\autoref{fig:prob2_xJ}。
由于$\sqrt{\frac{m}{k}}\frac{v}{l}=\frac{1}{4}$时系统没有完整运动周期,故未求出绝热不变量,未绘图。

\begin{figure}
    \centering
    \subfloat[$\sqrt{\frac{m}{k}}\frac{v}{l}=\frac{1}{16}$]{
        \includegraphics[width=0.48\textwidth]{figures/adiabatic_1.2_xJ16.pdf}
    }
    \subfloat[$\sqrt{\frac{m}{k}}\frac{v}{l}=\frac{1}{64}$]{
        \includegraphics[width=0.48\textwidth]{figures/adiabatic_1.2_xJ64.pdf}
    }\\
    \subfloat[$\sqrt{\frac{m}{k}}\frac{v}{l}=\frac{1}{256}$]{
        \includegraphics[width=0.7\textwidth]{figures/adiabatic_1.2_xJ256.pdf}
    }
    \caption{$x=2-vt$时的系统绝热不变量$\frac{J}{\sqrt{mk}l^2}$与$\xi = \frac{x}{l}$的关系。时间跨度为$\qty[0, \frac{2}{v}]$}
    \label{fig:prob2_xJ}
\end{figure}

可以发现,足够缓变时,系统的$J$是周期运动中的不变量。

\section{\texorpdfstring{$g$}{g}缓变}
$x$恒定为$0.2l$,$g$作如下周期性变化
\begin{equation}
    \frac{mg}{kl} = 2\cos(2\pi \sqrt{\frac{m}{k}}\nu\tau).
\end{equation}
$y(0) = -2l, p_y(0)=0$。

求解过程与 \autoref{sec:slow_x} 几乎完全一致,故不多赘述。
代码如下
{
    \linespread{1.0}
    \lstinputlisting[linerange=beg:prob_3-beg:prob_4]{1_adiabatic_invariant.cpp}
}

绘制相图于\autoref{fig:prob3_yp},绝热不变量的曲线于\autoref{fig:prob3_gJ}。
由于$\sqrt{\frac{m}{k}}\nu=\frac{1}{4}$及$\sqrt{\frac{m}{k}}\nu=\frac{1}{16}$时绝热不变量的数据点仅有0个与1个,故未绘图。

\begin{figure}
    \centering
    \subfloat[$\sqrt{\frac{m}{k}}\nu=\frac{1}{4}$]{
        \includegraphics[width=0.48\textwidth]{figures/adiabatic_1.3_yp4.pdf}
    }
    \subfloat[$\sqrt{\frac{m}{k}}\nu=\frac{1}{16}$]{
        \includegraphics[width=0.48\textwidth]{figures/adiabatic_1.3_yp16.pdf}
    }\\
    \subfloat[$\sqrt{\frac{m}{k}}\nu=\frac{1}{64}$]{
        \includegraphics[width=0.48\textwidth]{figures/adiabatic_1.3_yp64.pdf}
    }
    \subfloat[$\sqrt{\frac{m}{k}}\nu=\frac{1}{256}$]{
        \includegraphics[width=0.48\textwidth]{figures/adiabatic_1.3_yp256.pdf}
    }
    \caption{$\frac{mg}{kl} = 2\cos(2\pi \sqrt{\frac{m}{k}}\nu\tau)$时的系统相轨。时间跨度为$\qty[0, \frac{1}{2\nu}\sqrt{\frac{k}{m}}]$}
    \label{fig:prob3_yp}
\end{figure}

\begin{figure}
    \centering
    \subfloat[$\sqrt{\frac{m}{k}}\nu=\frac{1}{64}$]{
        \includegraphics[width=0.8\textwidth]{figures/adiabatic_1.3_gJ64.pdf}
    }\\
    \subfloat[$\sqrt{\frac{m}{k}}\nu=\frac{1}{256}$]{
        \includegraphics[width=0.8\textwidth]{figures/adiabatic_1.3_gJ256.pdf}
    }
    \caption{$\frac{mg}{kl} = 2\cos(2\pi \sqrt{\frac{m}{k}}\nu\tau)$时的系统绝热不变量$\frac{J}{\sqrt{mk}l^2}$与$\frac{mg}{kl}$的关系。时间跨度为$\qty[0, \frac{1}{2\nu}\sqrt{\frac{k}{m}}]$}
    \label{fig:prob3_gJ}
\end{figure}

对于参量变化较为剧烈的系统,无法看到绝热不变量的恒定特性。
而对于缓变系统,可以看到其在相图上从一个平衡位置跃变到另一个平衡位置,过程中一直有稳定的周期运动,绝热不变量在前后也是分别守恒的。

\section{贝瑞相位}

两参量$x, g$同时随时间变化,自$t_i$至$t_f$后参量复原。
设系统初末的相空间坐标分别为$(q_i, p_i), (q_f, p_f)$,定义贝瑞相位为
\begin{equation}
    \varphi_\mathrm{B} \equiv \qty[
        \omega(t_i) \int_{(q_i, p_i)}^{(q_f, p_f)} \dd{t}
        - \int_{t_i}^{t_f}\omega(t)\dd{t}
    ] \mod{2\pi}.
\end{equation}
其中,第一项指的是一个参量维持为$t_i$时的参量的系统演化到$(q_f, p_f)$的用时。
第二项指的是缓变系统各个瞬时的角频率$\omega(t)$的对时间积分。

让参量以$g(t)=\frac{2kl}{m}\cos(2\pi\nu t),x(t) = 2l\sin(2\pi\nu t)$演化,取初始$y=-2.1l,p_y=0$。
取$\sqrt{\frac{m}{k}}\nu=\frac{1}{256N}, N=1,2,3,4,5,6$,计算其贝瑞相位。

计算过程中,需要
\begin{enumerate}
    \item \label{enum:ivp}计算系统全程演化的IVP问题,对$p, y$关于$t$求插值,获得$(q_f, p_f)$;
    \item \label{enum:get_omega}对于一系列时间点$t_i$(可以比前面求解IVP时稀疏),利用该时刻的$p, y, x, g$,求解一个IVP问题。该假想系统仅需运行几个周期,以获得该处的$\omega(t)$;
    \item \label{enum:omega_spline}从上面离散的$\omega(t)$中对$t$插值,得到连续的$omega(t)$函数;
    \item 将上面的函数对时间积分,得到贝瑞相位的第二项;
    \item \label{enum:t0} 前面第 \ref{enum:get_omega} 步中,对于$t=t_i$时,要深入计算以获得贝瑞相位第一项。求解假想的静态系统从初态运行到$(q_f,p_f)$的用时。
\end{enumerate}

$\omega$对$t$插值时导数未知,故需要进行三次样条插值,需要编译时知道节点数量。
因此定义了函数模板\verb|solve_berry<N>|。

下面我们看一下函数
{
    \linespread{1.0}
    \lstinputlisting[linerange=beg:berry-step1]{1_adiabatic_invariant.cpp}
}
第 \ref{enum:ivp} 步,计算全程IVP,获得$(q_f, p_f)$。
时间步长为$0.1$。
{
    \linespread{1.0}
    \lstinputlisting[linerange=step1-step2]{1_adiabatic_invariant.cpp}
}
第 \ref{enum:get_omega} 步,求解离散的$t, \omega$。
这里,时间步长为$1.0$。
这里也顺便完成了第 \ref{enum:t0} 步,结果存入\verb|t0|。
{
    \linespread{1.0}
    \lstinputlisting[linerange=step2-step3]{1_adiabatic_invariant.cpp}
}
第 \ref{enum:omega_spline} 步,插值。
这里计算得到插值节点数量为\verb|256 * N|。
{
    \linespread{1.0}
    \lstinputlisting[linerange=step3-step4]{1_adiabatic_invariant.cpp}
}
最后计算相位。
{
    \linespread{1.0}
    \lstinputlisting[linerange=step4-step_output]{1_adiabatic_invariant.cpp}
}
输出
{
    \linespread{1.0}
    \lstinputlisting[linerange=step_output-end:berry]{1_adiabatic_invariant.cpp}
}

本小题对\verb|solve_berry()|的调用为
{
    \linespread{1.0}
    \lstinputlisting[linerange=beg:prob_4-beg:prob_5]{1_adiabatic_invariant.cpp}
}

可以得到相图于\autoref{fig:prob4_yp},角频率随时间的变化关系于\autoref{fig:prob4_t_omega}。
计算得到的贝瑞相位输出见\verb|adiabatic_1.4.txt|,如下
\begin{verbatim}
PROBLEM 4
===========================
t0 5.803633288226351	1/nu 256	phase -6.259686302557746
t0 5.289829875121671	1/nu 512	phase -6.271781027311363
t0 4.782929205787262	1/nu 768	phase -6.276693967590127
t0 4.279899964055827	1/nu 1024	phase -6.27767799648332
t0 3.776447393506435	1/nu 1280	phase -6.279071792438081
t0 3.273416102915363	1/nu 1536	phase -6.280047024163736
\end{verbatim}
列出了求得的$t_0$与$\varphi_\mathrm{B}$。
可以发现,这些相位都与$-2\pi$十分接近,近似不依赖于$\nu$。
\begin{figure}
    \centering
    \subfloat[$\sqrt{\frac{m}{k}}\nu=\frac{1}{256}$]{
        \includegraphics[width=0.48\textwidth]{figures/adiabatic_1.4_yp256.pdf}
    }
    \subfloat[$\sqrt{\frac{m}{k}}\nu=\frac{1}{512}$]{
        \includegraphics[width=0.48\textwidth]{figures/adiabatic_1.4_yp512.pdf}
    }\\
    \subfloat[$\sqrt{\frac{m}{k}}\nu=\frac{1}{768}$]{
        \includegraphics[width=0.48\textwidth]{figures/adiabatic_1.4_yp768.pdf}
    }
    \subfloat[$\sqrt{\frac{m}{k}}\nu=\frac{1}{1024}$]{
        \includegraphics[width=0.48\textwidth]{figures/adiabatic_1.4_yp1024.pdf}
    }\\
    \subfloat[$\sqrt{\frac{m}{k}}\nu=\frac{1}{1280}$]{
        \includegraphics[width=0.48\textwidth]{figures/adiabatic_1.4_yp1280.pdf}
    }
    \subfloat[$\sqrt{\frac{m}{k}}\nu=\frac{1}{1536}$]{
        \includegraphics[width=0.48\textwidth]{figures/adiabatic_1.4_yp1536.pdf}
    }
    \caption{$\frac{mg}{kl} = 2\cos(2\pi \sqrt{\frac{m}{k}}\nu\tau),\frac{x}{l} = 2\sin(2\pi \sqrt{\frac{m}{k}}\nu\tau)$时的系统相轨。时间跨度为$\qty[0, \frac{1}{\nu}\sqrt{\frac{k}{m}}]$}
    \label{fig:prob4_yp}
\end{figure}

\begin{figure}
    \centering
    \subfloat[$\sqrt{\frac{m}{k}}\nu=\frac{1}{256}$]{
        \includegraphics[width=0.48\textwidth]{figures/adiabatic_1.4_t_omega256.pdf}
    }
    \subfloat[$\sqrt{\frac{m}{k}}\nu=\frac{1}{512}$]{
        \includegraphics[width=0.48\textwidth]{figures/adiabatic_1.4_t_omega512.pdf}
    }\\
    \subfloat[$\sqrt{\frac{m}{k}}\nu=\frac{1}{768}$]{
        \includegraphics[width=0.48\textwidth]{figures/adiabatic_1.4_t_omega768.pdf}
    }
    \subfloat[$\sqrt{\frac{m}{k}}\nu=\frac{1}{1024}$]{
        \includegraphics[width=0.48\textwidth]{figures/adiabatic_1.4_t_omega1024.pdf}
    }\\
    \subfloat[$\sqrt{\frac{m}{k}}\nu=\frac{1}{1280}$]{
        \includegraphics[width=0.48\textwidth]{figures/adiabatic_1.4_t_omega1280.pdf}
    }
    \subfloat[$\sqrt{\frac{m}{k}}\nu=\frac{1}{1536}$]{
        \includegraphics[width=0.48\textwidth]{figures/adiabatic_1.4_t_omega1536.pdf}
    }
    \caption{$\frac{mg}{kl} = 2\cos(2\pi \sqrt{\frac{m}{k}}\nu\tau),\frac{x}{l} = 2\sin(2\pi \sqrt{\frac{m}{k}}\nu\tau)$时系统角频率随时间关系(已无量纲化)。时间跨度为$\qty[0, \frac{1}{\nu}\sqrt{\frac{k}{m}}]$}
    \label{fig:prob4_t_omega}
\end{figure}

改取$g(t)=\frac{0.3kl}{m}\cos(2\pi\nu t),x(t) = 0.3l\sin(2\pi\nu t)$演化,取初始$y=0.32l,p_y=0$。
仍取$\sqrt{\frac{m}{k}}\nu=\frac{1}{256N}, N=1,2,3,4,5,6$,计算其贝瑞相位。

本小题对\verb|solve_berry()|的调用为
{
    \linespread{1.0}
    \lstinputlisting[linerange=beg:prob_5-end:prob_5]{1_adiabatic_invariant.cpp}
}

可以得到相图于\autoref{fig:prob5_yp},角频率随时间的变化关系于\autoref{fig:prob5_t_omega}。
计算得到的贝瑞相位输出见\verb|adiabatic_1.5.txt|,如下
\begin{verbatim}
PROBLEM 5
===========================
t0 1.205210529032838	1/nu 256	phase -0.004786565168863888
t0 2.41922950912396	1/nu 512	phase -0.002688984207210865
t0 3.631519956940133	1/nu 768	phase -0.002024335805579369
t0 4.843573394670195	1/nu 1024	phase -0.001618307672259789
t0 6.055019456736868	1/nu 1280	phase -0.001897329569992223
t0 0.9837711992767286	1/nu 1536	phase -0.00178522224563693
\end{verbatim}
列出了求得的$t_0$与$\varphi_\mathrm{B}$。
可以发现,这些相位都与$0$十分接近,近似不依赖于$\nu$。
\begin{figure}
    \centering
    \subfloat[$\sqrt{\frac{m}{k}}\nu=\frac{1}{256}$]{
        \includegraphics[width=0.48\textwidth]{figures/adiabatic_1.5_yp256.pdf}
    }
    \subfloat[$\sqrt{\frac{m}{k}}\nu=\frac{1}{512}$]{
        \includegraphics[width=0.48\textwidth]{figures/adiabatic_1.5_yp512.pdf}
    }\\
    \subfloat[$\sqrt{\frac{m}{k}}\nu=\frac{1}{768}$]{
        \includegraphics[width=0.48\textwidth]{figures/adiabatic_1.5_yp768.pdf}
    }
    \subfloat[$\sqrt{\frac{m}{k}}\nu=\frac{1}{1024}$]{
        \includegraphics[width=0.48\textwidth]{figures/adiabatic_1.5_yp1024.pdf}
    }\\
    \subfloat[$\sqrt{\frac{m}{k}}\nu=\frac{1}{1280}$]{
        \includegraphics[width=0.48\textwidth]{figures/adiabatic_1.5_yp1280.pdf}
    }
    \subfloat[$\sqrt{\frac{m}{k}}\nu=\frac{1}{1536}$]{
        \includegraphics[width=0.48\textwidth]{figures/adiabatic_1.5_yp1536.pdf}
    }
    \caption{$\frac{mg}{kl} = 0.3\cos(2\pi \sqrt{\frac{m}{k}}\nu\tau),\frac{x}{l} = 0.3\sin(2\pi \sqrt{\frac{m}{k}}\nu\tau)$时的系统相轨。时间跨度为$\qty[0, \frac{1}{\nu}\sqrt{\frac{k}{m}}]$}
    \label{fig:prob5_yp}
\end{figure}

\begin{figure}
    \centering
    \subfloat[$\sqrt{\frac{m}{k}}\nu=\frac{1}{256}$]{
        \includegraphics[width=0.48\textwidth]{figures/adiabatic_1.5_t_omega256.pdf}
    }
    \subfloat[$\sqrt{\frac{m}{k}}\nu=\frac{1}{512}$]{
        \includegraphics[width=0.48\textwidth]{figures/adiabatic_1.5_t_omega512.pdf}
    }\\
    \subfloat[$\sqrt{\frac{m}{k}}\nu=\frac{1}{768}$]{
        \includegraphics[width=0.48\textwidth]{figures/adiabatic_1.5_t_omega768.pdf}
    }
    \subfloat[$\sqrt{\frac{m}{k}}\nu=\frac{1}{1024}$]{
        \includegraphics[width=0.48\textwidth]{figures/adiabatic_1.5_t_omega1024.pdf}
    }\\
    \subfloat[$\sqrt{\frac{m}{k}}\nu=\frac{1}{1280}$]{
        \includegraphics[width=0.48\textwidth]{figures/adiabatic_1.5_t_omega1280.pdf}
    }
    \subfloat[$\sqrt{\frac{m}{k}}\nu=\frac{1}{1536}$]{
        \includegraphics[width=0.48\textwidth]{figures/adiabatic_1.5_t_omega1536.pdf}
    }
    \caption{$\frac{mg}{kl} = 0.3\cos(2\pi \sqrt{\frac{m}{k}}\nu\tau),\frac{x}{l} = 0.3\sin(2\pi \sqrt{\frac{m}{k}}\nu\tau)$时系统角频率随时间关系(已无量纲化)。时间跨度为$\qty[0, \frac{1}{\nu}\sqrt{\frac{k}{m}}]$}
    \label{fig:prob5_t_omega}
\end{figure}

两种参量条件下,贝瑞相位都近似为$0$(即$-2\pi$)。

\end{document}
