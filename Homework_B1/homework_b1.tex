\documentclass[a4paper,unicode]{report}
\usepackage{amsmath}
\usepackage{amssymb}
\usepackage{textgreek}
\usepackage{esint}
\usepackage{subfig}
\usepackage{rotating}
\usepackage{booktabs}
\usepackage{longtable}
\usepackage{paralist}
\usepackage{titlesec}
\usepackage{graphicx}
\usepackage{physics}
\usepackage{ucs}
\usepackage{listings}
\usepackage{multirow}
\usepackage{tablefootnote}
\usepackage{indentfirst}
\usepackage{tikz}
\usepackage{warpcol}
\usepackage[pdftitle=计算物理第二次大作业第一部分,pdfauthor=***,bookmarksnumbered=true]{hyperref}
\usepackage{natbib}
\usepackage{xcolor}
\usepackage{xeCJK}
    \setCJKmainfont[BoldFont={Noto Serif CJK SC Bold},ItalicFont={FangSong}]{Noto Serif CJK SC}
    \setCJKsansfont[BoldFont={Noto Sans CJK SC Bold},ItalicFont={KaiTi}]{Noto Sans CJK SC}
    \setCJKmonofont[BoldFont={Noto Sans Mono CJK SC Bold}]{Noto Sans Mono CJK SC}

\title{计算物理第二次大作业第一部分}
\author{物理学院\quad ***\quad 1800011***}

\DeclareUnicodeCharacter{"00B0}{\textdegree}
\DeclareUnicodeCharacter{"00B5}{\textmu}

\renewcommand{\today}{\number\year 年\number\month 月\number\day 日}
\renewcommand{\refname}{参考文献}
\renewcommand{\abstractname}{摘要}
\renewcommand{\contentsname}{目录}
\renewcommand{\figurename}{图}
\renewcommand{\tablename}{表}
\renewcommand{\appendixname}{附录}
\renewcommand{\chaptername}{第}
\newcommand{\chapterendname}{章}

\renewcommand{\equationautorefname}{式}
\renewcommand{\figureautorefname}{图}
\newcommand{\subfigureautorefname}{子图}
\renewcommand{\tableautorefname}{表}
\renewcommand{\subsectionautorefname}{小节}
\newcommand{\mythmautorefname}{定理}
\renewcommand{\sectionautorefname}{\S}

\newtheorem{mythm}{定理}

\lstset{
    basicstyle=\scriptsize\ttfamily\color{black}, % print whole listing small
    keywordstyle=\color{teal}\bfseries,
    identifierstyle=, % nothing happens
    commentstyle=\color{gray}\ttfamily, % white comments
    stringstyle=\color{violet}, % typewriter type for strings
    showstringspaces=true,
    numbers=left,
    numberstyle=\tiny\color{brown},
    stepnumber=2,
    numbersep=5pt,
    firstnumber=auto,
    frame=lines,
    language=[11]C++,
    rangeprefix=/*,
    rangesuffix=*/,
    includerangemarker=false
}

% \pgfsetxvec{\pgfpoint{0.02cm}{0}}
% \pgfsetyvec{\pgfpoint{0}{0.1em}}

% \usetikzlibrary{datavisualization,plotmarks,datavisualization.formats.functions}
% \usetikzlibrary{math,fpu,datavisualization}
% \graphicspath{{figure/}}
\linespread{1.3}

\titleformat{\chapter}[display]{\huge\bfseries}{\chaptertitlename\ \thechapter\ \chapterendname}{20pt}{\Huge}

\begin{document}

\maketitle
\tableofcontents

\begin{center}
    \textbf{编译与运行环境说明}
\end{center}

本地操作系统版本为\textsf{Windows 10 x64 1909 (18363.815)}。

第二次大作业第一题主要使用\textsf{C++}编写,编译器为Windows下的\texttt{g++ 9.3.0} (Rev2, Built by MSYS2 project),标准为\textsf{C++17}。
除标准库外未使用其他第三方库。
最后上交的程序的编译命令为\begin{verbatim}
    g++ -O2 -static -std=c++17 -o <可执行文件名> <源文件名>
\end{verbatim}

对于绘图部分,使用了必要的\textsf{Python}的第三方库\texttt{matplotlib}进行处理。

% \setcounter{chapter}{2}
\chapter{正交多项式}
在区间$(a, b)$上可以定义实函数$f,g$的内积
\begin{equation}
    \braket{f}{g} = \int_a^b f(x)g(x)\rho(x)\dd{x},
\end{equation}
其中$\rho(x)$为权函数。
可以定义区间$(a, b)$上在权函数$\rho(x)$下平方可积函数空间$\mathcal{L}^2(a, b)$上的正交归一多项式$\{Q_n(x)\}$,满足
\begin{align}
    \braket{Q_i}{Q_j} &{}= \delta_{ij}, \label{eq:ortho_normal}\\
    \braket{Q_n}{x^m} &{}= 0,\quad\forall\ m<n. \label{eq:normal_to_power}
\end{align}
依题意,所有的正交多项式都以内积来确定归一化常数。

获得正交多项式的方法可以是将线性无关的函数组$\{x^n\}$进行Gram--Schmidt正交化。
取迭代起点为常函数$Q_0(x) = \frac{1}{\sqrt{\braket{1}}}$,则在已知$Q_{0},Q_1,\cdots,Q_n$后,$Q_{n+1}$可以这样获得:
\begin{align}
    \gamma_{i, n} &{}= \braket{Q_i}{xQ_n},\quad i=0,1,\cdots,n,\label{eq:def_gamma}\\
    V_n(x) &{}= xQ_n(x) - \sum_{i=0}^n \gamma_{i,n} Q_i(x), \label{eq:V_from_Q}\\
    \alpha_n &{}= \sqrt{\braket{V_n}},\label{eq:alpha}\\
    Q_{n+1}(x) &{}= \frac{V_n(x)}{\alpha_n}.\label{eq:normal}
\end{align}

\section{多项式递归算法的合理性}
\subsection{正交归一性}
\paragraph{归一性}
显然,根据定义,$Q_0(x)$归一。
而其他$Q_i$由递推式导出,根据式 \eqref{eq:alpha} \eqref{eq:normal},其归一性质是显然的。

\paragraph{各正交函数间的正交性}
即证式 \eqref{eq:ortho_normal}。

使用数学归纳法。
假设$\qty{Q_{0},Q_1,\cdots,Q_k}$已是正交归一的,下面证明$Q_{k+1}$与它们正交归一。
在式 \eqref{eq:normal} 两侧作与$Q_i\ (i \le k)$的内积,则
\begin{multline}
    \braket{Q_i}{Q_{k+1}}
    = \frac{1}{\alpha_k}\braket{Q_i}{V_k}
    = \frac{1}{\alpha_k}\qty(\gamma_{i, k} - \sum_{j=0}^k \gamma_{j,k} \braket{Q_i}{Q_j}) \\
    = \frac{1}{\alpha_k}\qty(\gamma_{i, k} - \sum_{j=0}^k \gamma_{j,k} \delta_{ij})
    = \frac{1}{\alpha_k}\qty(\gamma_{i, k} - \gamma_{i,k}) = 0.
\end{multline}
因此,$\qty{Q_{0},Q_1,\cdots,Q_{k+1}}$也是正交归一的。

而$\qty{Q_{0}}$显然由定义也是正交归一的。
根据数学归纳法原理,$\qty{Q_n}$是一组正交归一函数,式 \eqref{eq:ortho_normal} 得证。

\paragraph{\texorpdfstring{$Q_n$}{Q_n}与所有幂次小于\texorpdfstring{$n$}{n}的幂函数正交}
即证式 \eqref{eq:normal_to_power}。

先证明$Q_n$为$n$次多项式。
由定义,$Q_0$为$0$次多项式。

假设$\qty{Q_{0},Q_1,\cdots,Q_k}$分别是$0,1,\cdots,k$次多项式。
根据式 \eqref{eq:V_from_Q},$V_k$为$k+1$次多项式,从而根据式 \eqref{eq:normal},$Q_{k+1}$为$k+1$次多项式。
由数学归纳法,$Q_n$为$n$次多项式。

因此,我们可以写出
\begin{equation}
    \mqty(
        Q_0\\   Q_1\\   \vdots\\    Q_n
    )
    = \mqty(
        a_{00}&&&\\
        a_{10}& a_{11}&&\\
        \vdots& \vdots& \ddots&\\
        a_{n0}&  a_{n1}& \cdots& a_{nn}
    ) \mqty(
        1\\   x\\   \vdots\\    x^n
    ),
\end{equation}
或反解为
\begin{equation}
    \mqty(
        1\\   x\\   \vdots\\    x^n
    )
    = \mqty(
        b_{00}&&&\\
        b_{10}& b_{11}&&\\
        \vdots& \vdots& \ddots&\\
        b_{n0}&  b_{n1}& \cdots& b_{nn}
    ) \mqty(
        Q_0\\   Q_1\\   \vdots\\    Q_n
    ).
\end{equation}

因此,
\begin{equation}
    x^m = \sum_{i=0}^m b_{mi} Q_i,
\end{equation}
根据式 \eqref{eq:ortho_normal},$m < n$时,
\begin{equation}
    \braket{Q_n}{x^m} = \sum_{i=0}^m b_{mi} \braket{Q_n}{Q_i} = \sum_{i=0}^m b_{mi} \delta_{ni} = 0,
\end{equation}
式 \eqref{eq:normal_to_power} 得证。
而$m\ge n$时,
\begin{equation} \label{eq:inner_product_Q_power}
    \braket{Q_n}{x^m} = \sum_{i=0}^m b_{mi} \braket{Q_n}{Q_i} = \sum_{i=0}^m b_{mi} \delta_{ni} = b_{mn},
\end{equation}

\subsection{系数\texorpdfstring{$\gamma_{i,n}$}{γ\_\{i,n\}}的简化}
\label{ssec:gamma_i_n}
对特定的$n$,使用数学归纳法证明
\begin{equation} \label{eq:gamma_eq_0}
    \gamma_{i, n} = 0,\quad i=0,1,\cdots,n-2.
\end{equation}

首先证明$i = 0$的情况。
式 \eqref{eq:V_from_Q} 即
\begin{equation}\label{eq:V_from_Q_b}
    \alpha_n Q_{n+1}(x) = xQ_n(x) - \sum_{i=0}^n \gamma_{i,n} Q_i(x). \tag{\ref{eq:V_from_Q}b}
\end{equation}
将式 \eqref{eq:V_from_Q_b} $(n \ge 2)$ 两侧与$x^0$作内积,得到
\begin{multline}
    \alpha_n \braket{x^0}{Q_{n+1}(x)} = 0 \\
    = \braket{x}{Q_n(x)} - \sum_{i=0}^n \gamma_{i,n} \braket{x^0}{Q_i(x)} = 0 - \gamma_{0,n} \braket{x^0}{Q_0(x)}.
\end{multline}
根据式 \eqref{eq:inner_product_Q_power},$\braket{x^0}{Q_0(x)}$非零,故$\gamma_{0,n}=0$。
式 \eqref{eq:V_from_Q} 变为
\begin{equation}\label{eq:V_from_Q_c}
    \alpha_n Q_{n+1}(x) = xQ_n(x) - \sum_{i=1}^n \gamma_{i,n} Q_i(x). \tag{\ref{eq:V_from_Q}c}
\end{equation}

再假设对于$i=0,1,\cdots,k; k<n-2 $,有$\gamma_{i, n} = 0$,那么式 \eqref{eq:V_from_Q} 变为
\begin{equation}\label{eq:V_from_Q_d}
    \alpha_n Q_{n+1}(x) = xQ_n(x) - \sum_{i=k+1}^n \gamma_{i,n} Q_i(x). \tag{\ref{eq:V_from_Q}d}
\end{equation}
将式 \eqref{eq:V_from_Q_d} 两侧与$x^{k+1}$作内积,得到
\begin{multline}
    \alpha_n \braket{x^{k+1}}{Q_{n+1}(x)} = 0 \\
    = \braket{x^{k+2}}{Q_n(x)} - \sum_{i=k+1}^n \gamma_{i,n} \braket{x^{k+1}}{Q_i(x)} \\
    = 0 - \gamma_{k+1,n} \braket{x^{k+1}}{Q_{k+1}(x)}.
\end{multline}
故$\gamma_{k+1,n}=0$。

下面证明前述的递推关系在$k = n-2$时截止。
将式 \eqref{eq:V_from_Q_d} 两侧与$x^{k+1} = x^{n-1}$作内积,得到
\begin{multline}
    \alpha_n \braket{x^{n-1}}{Q_{n+1}(x)} = 0 \\
    = \braket{x^{n}}{Q_n(x)} - \sum_{i=n-1}^n \gamma_{i,n} \braket{x^{n-1}}{Q_i(x)} \\
    = \braket{x^{n}}{Q_n(x)} - \gamma_{n-1,n} \braket{x^{n-1}}{Q_{n-1}(x)}.
\end{multline}
即有
\begin{equation}
    \gamma_{n-1,n} = \frac{\braket{x^{n}}{Q_n(x)}}{\braket{x^{n-1}}{Q_{n-1}(x)}} \neq 0.
\end{equation}
所以由数学归纳法,式 \eqref{eq:gamma_eq_0} 成立。

\subsection{系数\texorpdfstring{$\alpha_{n-1}$}{α\_\{n-1\}}与\texorpdfstring{$\gamma_{n-1,n}$}{γ\_\{n-1,n\}}的关系}
由\autoref{ssec:gamma_i_n} 的结论,
\begin{equation} \label{eq:recurrence}
    \alpha_n Q_{n+1}(x) = \qty(x - \gamma_{n,n}) Q_n(x) - \gamma_{n-1,n} Q_{n-1}(x).
\end{equation}
两边均与$Q_{n+1}$作内积,利用正交归一性,可得
\begin{equation}
    \alpha_n = \braket{Q_{n+1}}{xQ_n} = \braket{Q_n}{xQ_{n+1}}.
\end{equation}

与式 \eqref{eq:def_gamma} 对比,可以得到
\begin{equation}
    \alpha_n = \gamma_{n, n+1},\quad\text{即}\ \alpha_{n-1} = \gamma_{n-1, n}.
\end{equation}

\section{利用递推关系计算一些多项式}
\subsection{求勒让德多项式与拉盖尔多项式的一些值}
记$\beta_n = \gamma_{n, n}$,那么递推关系 \eqref{eq:recurrence} 可写成
\begin{equation} \label{eq:beau_recurrence}
    \alpha_n Q_{n+1}(x) = \qty(x - \beta_n) Q_n(x) - \alpha_{n-1} Q_{n-1}(x).
\end{equation}
下面利用这个关系求$n=2,\ 16,\ 128,\ 1024$阶的勒让德多项式$\mathrm{P}_n(x)$与拉盖尔多项式$\mathrm{L}_n(x)$在$x=0.5$处的函数值。

\subsubsection{勒让德多项式}
通常定义的勒让德函数满足
\begin{gather}
    (n+1)\mathrm{P}_{n+1}=(2n+1)x\mathrm{P}_{n}-n\mathrm{P}_{n-1},\\
    \mathrm{P}_0 = 1,\quad \mathrm{P}_1 = x.
\end{gather}
归一化后的勒让德函数满足
\begin{equation}
    \hat{\mathrm{P}}_n = \sqrt{n + \frac{1}{2}} \mathrm{P}_n.
\end{equation}
那么,
\begin{gather}
    (n+1)\frac{\hat{\mathrm{P}}_{n+1}}{\sqrt{(n+1)+1/2}}=x(2n+1)\frac{\hat{\mathrm{P}}_{n}}{\sqrt{n+1/2}}-n\frac{\hat{\mathrm{P}}_{n-1}}{\sqrt{(n-1)+1/2}},\\
    \frac{\hat{\mathrm{P}}_{0}}{\sqrt{0+1/2}} = 1,\quad \frac{\hat{\mathrm{P}}_{1}}{\sqrt{1+1/2}} = x.
\end{gather}

因此函数可实现为
{
    \linespread{1.0}
    \lstinputlisting[linerange=beg:Pn_dec-end:Pn_dec]{1_Orthogonal_Polynomials.cpp}
    \lstinputlisting[linerange=beg:Pn-end:Pn]{1_Orthogonal_Polynomials.cpp}
}

为减少开方运算,函数中递推得到了$\frac{\hat{\mathrm{P}}_{n}}{\sqrt{n+1/2}}$,在最后乘上$\sqrt{n+1/2}$。
另外,对于$n=0, 1$的特殊情况进行了简化处理。

在函数\texttt{main()}中调用该函数计算要求的值,如下
{
    \linespread{1.0}
    \lstinputlisting[linerange=beg:prob2.1_P-end:prob2.1_P]{1_Orthogonal_Polynomials.cpp}
}
程序输出为
\begin{verbatim}
==========================
PROBLEM 2.1
--------------------------
Legendre polynomial P_n(x) at x=0.5
n=2,    P_n(x)=-0.1976423537605237
n=16,   P_n(x)=-0.6087144378627884
n=128,  P_n(x)=-0.2214407352688623
n=1024, P_n(x)=-0.6063038194703785
\end{verbatim}

% \appendix
% \renewcommand\chapterendname{}

\subsubsection{拉盖尔多项式}
通常定义的拉盖尔函数已是归一化的,满足
\begin{gather}
    (n+1)\mathrm{L}_{n+1}=(2n+1-x)\mathrm{L}_{n}-n\mathrm{L}_{n-1},\\
    \mathrm{L}_0 = 1,\quad \mathrm{L}_1 = -x+1.
\end{gather}

因此函数可实现为
{
    \linespread{1.0}
    \lstinputlisting[linerange=beg:Ln_dec-end:Ln_dec]{1_Orthogonal_Polynomials.cpp}
    \lstinputlisting[linerange=beg:Ln-end:Ln]{1_Orthogonal_Polynomials.cpp}
}

其中对于$n=0, 1$的特殊情况进行了简化处理。

在函数\texttt{main()}中调用该函数计算要求的值,如下
{
    \linespread{1.0}
    \lstinputlisting[linerange=end:prob2.1_P-end:prob2.1]{1_Orthogonal_Polynomials.cpp}
}
程序输出为
\begin{verbatim}
--------------------------
Laguerre polynomial L_n(x) at x=0.5
n=2,    L_n(x)=0.125
n=16,   L_n(x)=0.09265564418598525
n=128,  L_n(x)=-0.2278406908864269
n=1024, L_n(x)=0.1340328657018783
\end{verbatim}

\subsection{求厄密多项式与切比雪夫多项式的一些零点}
式 \eqref{eq:beau_recurrence} 可以写为矩阵形式,
\begin{equation}\label{eq:mat_eig}
    x \mqty(
        Q_0\\   Q_1\\   Q_2\\   \vdots\\    Q_{n-1}
    )
    = \mqty(
        \beta_0&    \alpha_0&&&\\
        \alpha_0& \beta_1&  \alpha_1&&\\
        &   \alpha_1&   \beta_2&    \ddots&\\
        && \ddots&\ddots&\\
        &&&&    \beta_{n-1}
    ) \mqty(
        Q_0\\   Q_1\\   Q_2\\   \vdots\\    Q_{n-1}
    )
    + \mqty(
        0\\   0\\   0\\   \vdots\\    \alpha_{n-1}Q_n
    ).
\end{equation}

可以发现,$Q_n$的$n$个零点恰为三对角系数矩阵的$n$个本征值。
下面用QR算法和二分法求解实对称矩阵本征值问题。
由于本题求解的已是海森堡矩阵,故略去了使用Householder变换进行海森堡化的部分。

\subsubsection{QR算法}
实对称矩阵的QR算法的底层实现是\texttt{misc/eigvals.h}中的\texttt{\_eig\_hessenberg\_symm\_qr\_()}函数,如下。
其对从\texttt{start}到\texttt{end}间的子矩阵进行QR迭代对角化。
其可以在非空的指针\texttt{p\_q\_mat}指向的矩阵右方作用本次得到的正交矩阵。
{
    \linespread{1.0}
    \lstinputlisting[linerange=beg:symm_qr-end:symm_qr]{../misc/eigvals.h}
}
向量\texttt{zero\_break}在计算前扫描为0的非对角元,将矩阵彻底块对角化。
\texttt{mat\_range}存储当前正在计算的子三对角矩阵的最右下角元素编号;\texttt{lower\_bound}存储当前正在计算的子三对角矩阵的最左上角元素编号,为不断从\texttt{zero\_break}末尾读取得到。

在\texttt{mat\_range}未执行到\texttt{start}时,保持循环。
在循环开头检查有没有完成对子三对角矩阵的对角化,如已完成,则更新\texttt{lower\_bound}。

计算Wilkinson shift,即当前三对角矩阵最右下角的$2\times 2$矩阵的两个本征值中最接近最右下角元素的一个。

随后做初始化。
根据隐式QR定理,只要计算了正交矩阵$\vb{Q}$的第一列,其后的各列均被唯一确定了。
因此,使用Givens变换确定$\vb{Q}$,使得最左上方的非对角元为0。
但将对称的正交矩阵载作用后,除原有的三对角部分外,$(2,1)$处也出现了非零元素。
将其存入\texttt{bulge}。

接下来的主循环中,通过Givens变换,将\texttt{bulge}逐步向矩阵的右下赶,直到重新获得海森堡矩阵。
这时,我们完成了一次QR迭代。

当非对角元足够小(判据\texttt{criteria})时,将其取为零,并缩减矩阵规模。

\subsubsection{二分法}
对于对称三对角矩阵$\vb{T}$,$\vb{T}-\lambda\vb{I}$的$i$阶顺序主子式组成的序列$\qty{p_i(\lambda)}$称为其\textbf{Sturm序列}。
这一序列的变号数$s_n(\lambda)$恰为该矩阵在区间$(-\infty, \lambda)$内特征值的个数。
利用这一性质,可以用二分法求解本征值,且可以任意指定要求的本征值是升序的第几个。

实对称矩阵的二分法的底层实现是\texttt{misc/eigvals.h}中的\texttt{\_eig\_hessenberg\_symm\_bisection()}函数,如下。
其返回所有本征值升序排列组成的数组。
{
    \linespread{1.0}
    \lstinputlisting[linerange=beg:symm_bis-end:symm_bis]{../misc/eigvals.h}
}

首先定义了计算变号数的闭包表达式\texttt{sturm\_change\_of\_sign}。
按原始定义,有递推公式
\begin{gather}
    p_0 = 1,\quad p_1 = a_1 - \lambda,\\
    p_i = (a_i - \lambda)p_{i-1} - b_i^2p_{i-2},\quad i=2,\cdots,n.
\end{gather}
其中,$a_i$为$\vb{T}$从1开始编号的对角元,$b_i$为$\vb{T}$从2开始编号的次对角元。
这一递推公式数值不稳定,改用$q_i = \frac{p_i}{p_{i-1}}$计算变号数。
若$q_i=0$,则定义为此次未变号,且必有$q_{i+1} = -\infty$。
程序中对此进行了判断。

易得,矩阵本征值的绝对值的上界为矩阵的无穷范数。
故计算了无穷范数作为二分法的左右边界。

最后的循环中,依次根据变号数进行二分查找。
每次均利用上次查找结果缩小搜索范围。
将结果存入数组\texttt{res}。

\subsubsection{厄密多项式零点}
归一化的厄密多项式满足
\begin{equation}
    x\mathrm{H}_n(x) = \sqrt{\frac{n+1}{2}}\mathrm{H}_{n+1}(x) + \sqrt{\frac{n}{2}}\mathrm{H}_{n-1}(x).
\end{equation}
因此,
\begin{equation}
    \alpha_i = \sqrt{\frac{i+1}{2}},\quad \beta_i = 0.
\end{equation}
这一矩阵十分容易的可用如下函数生成:
{
    \linespread{1.0}
    \lstinputlisting[linerange=beg:hermite_dec-end:hermite_dec]{1_Orthogonal_Polynomials.cpp}
    \lstinputlisting[linerange=beg:hermite-end:hermite]{1_Orthogonal_Polynomials.cpp}
}
在\texttt{main()}中用两种方法计算待求本征值,代码为
{
    \linespread{1.0}
    \lstinputlisting[linerange=end:prob2.1-end:prob2.2_hermite]{1_Orthogonal_Polynomials.cpp}
}
输出为
{
\footnotesize
\begin{verbatim}
==========================
PROBLEM 2.2
--------------------------
j'th zero of Hermite polynomial of degree 1024 H_n(x)
j=      2                       16                      128                     1024
QR      -44.35362564023855      -41.29744814312823      -28.75853394023516      44.7445685115969
bisec   -44.3536256402386       -41.29744814312797      -28.75853394023522      44.7445685115968
\end{verbatim}
}
与\textsf{Python}的第三方库\texttt{scipy.special}中的结果进行对比,二分法结果完全一致,QR算法在\texttt{double}的最后一位精度上有小差异。

\subsubsection{切比雪夫多项式零点}
归一化的切比雪夫多项式满足
\begin{gather}
    x\mathrm{T}_0 = \frac{\mathrm{T}_1}{\sqrt{2}},\quad x\mathrm{T}_1 = \frac{\mathrm{T}_0}{\sqrt{2}} + \frac{\mathrm{T}_2}{2},\\
    x\mathrm{T}_n = \frac{\mathrm{T}_{n+1} + \mathrm{T}_{n-1}}{2}.
\end{gather}
因此,
\begin{equation}
    \alpha_0 = \frac{1}{\sqrt{2}},\quad\alpha_i = \frac{1}{2},\quad \beta_n = 0.
\end{equation}
这一矩阵十分容易的可用如下函数生成:
{
    \linespread{1.0}
    \lstinputlisting[linerange=beg:chebyshev_dec-end:chebyshev_dec]{1_Orthogonal_Polynomials.cpp}
    \lstinputlisting[linerange=beg:chebyshev-end:chebyshev]{1_Orthogonal_Polynomials.cpp}
}
在\texttt{main()}中用两种方法计算待求本征值,代码为
{
    \linespread{1.0}
    \lstinputlisting[linerange=end:prob2.2_hermite-end:prob2.2]{1_Orthogonal_Polynomials.cpp}
}
输出为
{
\footnotesize
\begin{verbatim}
--------------------------
j'th zero of Chebyshev polynomial of degree 1024 T_n(x)
j=      2                       16                      128                     1024
QR      -0.9999894110819281     -0.9988695499142821     -0.9244654743252627     0.9999988234517019
bisec   -0.9999894110819283     -0.9988695499142836     -0.9244654743252627     0.9999988234517018
\end{verbatim}
}
已知$\mathrm{T}_n(x)$的第$k$个零点为$\cos(\frac{2(n-k)-1}{2n}\pi)$。
对比可以发现,二分法结果在\texttt{double}精度内完全正确,QR法最后一位有小偏差。

\subsubsection{算法精度与用时对比}
\paragraph{精度}
前面已经对比过了。
二分法的精度由二分区间宽度决定,这里设为了机器精度$\sim 10^{-16}$。
QR算法的精度则由将非对角元设为0的判据决定,这里设为了机器精度的根号$\sim 10^{-8}$。
两者最后的结果都非常好,二分法在浮点数意义下精确,而QR算法的偏差也仅在最后一位有效位上。

\paragraph{用时}
方面,在\texttt{main()}函数中加入如下内容,如果命令行传入参数\texttt{t}则进行用时测试。
{
    \linespread{1.0}
    \lstinputlisting[linerange=end:prob2.2-end:timing]{1_Orthogonal_Polynomials.cpp}
}
对较快的QR算法进行50次重复,对二分法重复5次。
最后输出单次运行耗时(单位\textmu{}s)。

在编译命令为
\begin{verbatim}
g++ -std=c++17 -o <可执行文件名> <源文件名>
\end{verbatim}
同时计算机接通电源的情况下,计时输出为
\begin{verbatim}
--------------------------
TIMING Chebyshev * 50 for QR and * 5 for bisection
QR (mus per run)        113851.76
bisec (mus per run)     1868566.8
\end{verbatim}

QR算法单次运行约0.11 s,二分法则为1.7 s,相差约16倍。

编译进行优化后
\begin{verbatim}
g++ -O2 -std=c++17 -o <可执行文件名> <源文件名>
\end{verbatim}
计时输出为
\begin{verbatim}
--------------------------
TIMING Chebyshev * 50 for QR and * 5 for bisection
QR (mus per run)        19537.1
bisec (mus per run)     349787.2
\end{verbatim}
QR算法单次运行约0.020 s,二分法则为0.35 s,相差约18倍。

所以,在需要计算所有本征值时,虽然二分法精度更高,但QR算法快得多。
然而,本题其实只需要求4个本征值。
如果仅使用二分法计算4个本征值的话,那么其用时将会大概缩减为$\frac{1}{256}$。
这样的话,其将比QR法快出几十倍。

\subsection{插值多项式}
\subsubsection{获得插值多项式}
点系$\qty{x_i}$上的插值多项式$\qty{f_j(x)}$应满足
\begin{equation}
    f_j(x_i) = \delta_{ij}.
\end{equation}
下面我们尝试从前面得到的归一正交多项式$\qty{Q_i}$中构造插值多项式。

由矩阵方程 \eqref{eq:mat_eig},可以发现,$Q_n$的$n$个零点恰为矩阵的$n$个本征值。
设$x_j$为第$j$个零点,那么,其本征矢为
\begin{equation}
    \mqty(
        Q_0(x_j)&   Q_1(x_j)&   Q_2(x_j)&   \cdots&    Q_{n-1}(x_j)
    )^T.
\end{equation}

对于实对称矩阵,其所有本征矢正交,因此
\begin{multline}
    \mqty(
        Q_0(x_j)&   Q_1(x_j)&   \cdots&    Q_{n-1}(x_j)
    )^T\mqty(
        Q_0(x_i)&   Q_1(x_i)&   \cdots&    Q_{n-1}(x_i)
    ) \\
    = \sum_{k=0}^{n-1} Q_k(x_j)Q_k(x_i) = w_j\delta_{ji},
\end{multline}
若定义
\begin{equation}
    f_j(x) := \sum_{k=0}^{n-1} Q_k(x_j)Q_k(x),
\end{equation}
那么它满足
\begin{equation} \label{eq:interpolation_poly}
    f_j(x_i) = w_j\delta_{ji},
\end{equation}
是区间上的插值多项式。

也可用另一种方式证明。
记式 \eqref{eq:mat_eig} 为
\begin{equation}\label{eq:mat_eig_simple}
    x\vb{Q}(x) = \vb{T} \vb{Q}(x) + \alpha_{n-1}Q_n \vu{e}_{n-1},
\end{equation}
则
\begin{gather}
    x\vb{Q}(y)^T\vb{Q}(x) = \vb{Q}(y)^T\vb{T} \vb{Q}(x) + \alpha_{n-1}Q_n \vb{Q}(y)^T\vu{e}_{n-1},\\
    y\vb{Q}(x)^T\vb{Q}(y) = \vb{Q}(x)^T\vb{T} \vb{Q}(y) + \alpha_{n-1}Q_n \vb{Q}(x)^T\vu{e}_{n-1},
\end{gather}
即(由于所有元素都是实的,而且矩阵对称)
\begin{equation}
    x\vb{Q}(y)^T\vb{Q}(x) - \alpha_{n-1}Q_n \vb{Q}(y)^T\vu{e}_{n-1} = y\vb{Q}(x)^T\vb{Q}(y) - \alpha_{n-1}Q_n \vb{Q}(x)^T\vu{e}_{n-1},
\end{equation}
或
\begin{equation} \label{eq:simple_fj}
    \sum_{k=0}^{n-1} Q_k(x)Q_k(y) = \alpha_{n-1} \frac{Q_{n-1}(y)Q_n(x) - Q_n(y)Q_{n-1}(x)}{x-y}.
\end{equation}
显然可以从上式直接得到 \eqref{eq:interpolation_poly}$i\neq j$的部分。
取$y\to x$,可以得到
\begin{equation}
    \sum_{k=0}^{n-1} Q_k(x)Q_k(x) = \alpha_{n-1} \qty(Q_{n-1}(x)Q'_n(x) - Q_n(x)Q'_{n-1}(x)) \neq 0,
\end{equation}
即式 \eqref{eq:interpolation_poly} 中的系数
\begin{gather}
    \begin{aligned}
        w_j &{}=\alpha_{n-1} \qty(Q_{n-1}(x_j)Q'_n(x_j) - Q_n(x_j)Q'_{n-1}(x_j)),\\
        &{} = \alpha_{n-1} Q_{n-1}(x_j)Q'_n(x_j),
    \end{aligned}\\
    \begin{aligned}
        f_j(x_i) &{}= \sum_{k=0}^{n-1} Q_k(x_j)Q_k(x_i) \\
        &{}= \delta_{ji}\alpha_{n-1} Q_{n-1}(x_j)Q'_n(x_j).
    \end{aligned}
\end{gather}

\subsubsection{高斯积分权值}
带权的插值求积公式为
\begin{equation}
    \int_a^b\rho(x)f(x)\dd{x} \approx \sum_{k=0}^{n-1} A_kf(x_k),
\end{equation}
同时,我们也已经知道,$\qty{x_k}$为$Q_n(x)$的$n$个零点。
那么为了求积分权重$A_j$,我们可以取$f(x) = f_j(x)$,即插值多项式,那么
\begin{equation}
    \int_a^b\rho(x)f_j(x)\dd{x} \approx \sum_{k=0}^{n-1} A_kf_j(x_k) = A_jf_j(x_j) = A_jw_j,
\end{equation}
展开$f_j(x)$,得到
\begin{multline}
    \sum_{k=0}^{n-1} Q_k(x_j) \int_a^b\rho(x)Q_k(x)\dd{x}
    = \sum_{k=0}^{n-1} Q_k(x_j) \braket{Q_k}{1}\\
    = Q_0(x_j) \braket{Q_0}{1}
    = A_j \alpha_{n-1} \qty(Q_{n-1}(x_j)Q'_n(x_j) - Q_n(x_j)Q'_{n-1}(x_j)).
\end{multline}

而根据定义,
\begin{equation}
    Q_0 = \frac{1}{\sqrt{\braket{1}}},\quad Q_0 \braket{Q_0}{1} = 1,
\end{equation}
故系数
\begin{multline}\label{eq:gauss_coef}
    A_j = \frac{1}{w_j} = \frac{1}{\sum_{k=0}^{n-1} Q_k(x_j)Q_k(x_j)} \\
    = \frac{1}{\alpha_{n-1} \qty(Q_{n-1}(x_j)Q'_n(x_j) - Q_n(x_j)Q'_{n-1}(x_j))}\\
    = \frac{1}{\alpha_{n-1} Q_{n-1}(x_j)Q'_n(x_j)}.
\end{multline}

\subsubsection{厄密多项式}
先定义一个通用的5点数值微分函数,选取步长为$2^{-25}$。
{
    \linespread{1.0}
    \lstinputlisting[linerange=beg:deriv-end:deriv]{1_Orthogonal_Polynomials.cpp}
}
其他求导函数将直接调用它。
然后定义厄密多项式函数
{
    \linespread{1.0}
    \lstinputlisting[linerange=beg:hermite_dec-end:hermite_dec]{1_Orthogonal_Polynomials.cpp}
    \lstinputlisting[linerange=beg:hermite-end:hermite]{1_Orthogonal_Polynomials.cpp}
}
递归方式与前面类似,不再赘述。

根据式 \eqref{eq:gauss_coef},在\texttt{main()}中定义lambda函数\texttt{func\_hermite()}求权系数。
{
    \linespread{1.0}
    \lstinputlisting[linerange=end:timing-end:prob2.3_hermite]{1_Orthogonal_Polynomials.cpp}
}
然而由于厄密多项式在无穷远处的奇异性质,均发生了发散。
输出为
{
\footnotesize
\begin{verbatim}
==========================
PROBLEM 2.3
--------------------------
j'th Gaussian integration weight of Hermite polynomial of degree 1024 H_n(x)
j=      2                       16                      128                     1024
        nan     nan     0       nan
\end{verbatim}
}

为避免其发散,尝试使用权函数$\rho(x)$。
定义带权的厄密多项式$\sqrt{\rho(x)}\mathrm{H}_n(x) = \exp(-x^2/2)\mathrm{H}_n(x)$。
注意到,在我们求值的地方(如$x=-44.353\,625\,640\,238\,6$),权函数其实发生了下溢。
因此,我们将权函数在迭代过程中逐步乘上去:
{
    \linespread{1.0}
    \lstinputlisting[linerange=beg:hermite_weighted_dec-end:hermite_weighted_dec]{1_Orthogonal_Polynomials.cpp}
    \lstinputlisting[linerange=beg:hermite_weighted-end:hermite_weighted]{1_Orthogonal_Polynomials.cpp}
}
同时,导数的定义也需要修改。
\begin{equation}
    e^{-\frac{x^2}{2}}\dv{\mathrm{H}_n}{x} = xe^{-\frac{x^2}{2}}\mathrm{H}_n + \dv{x} e^{-\frac{x^2}{2}}\mathrm{H}_n,
\end{equation}
可有
{
    \linespread{1.0}
    \lstinputlisting[linerange=beg:hermite_deri_weighted-end:hermite_deri_weighted]{1_Orthogonal_Polynomials.cpp}
}

\texttt{main()}函数部分如下,成功工作。
\texttt{func\_hermite()}求权系数。
{
    \linespread{1.0}
    \lstinputlisting[linerange=end:prob2.3_hermite-end:prob2.3_hermite_weighted]{1_Orthogonal_Polynomials.cpp}
}
输出为
{
\footnotesize
\begin{verbatim}
--------------------------
j'th Gaussian integration weight times rho(x) of Hermite polynomial of degree 1024 H_n(x)
j=      2                       16                      128                     1024
        0.3466901837717924      0.1694966208311146      0.08987182406046013     0.4539826520829535
\end{verbatim}
}
注意到,\textbf{这些高斯积分系数$A_j$是已经被扩大了$\exp(x_j^2)$倍},其本身是极小的,会导致双精度浮点数下溢。

根据式 \eqref{eq:simple_fj},
\begin{equation}
    f_j(x) = \alpha_{n-1} \frac{\mathrm{H}_{n-1}(x_j)\mathrm{H}_n(x)}{x - x_j},
\end{equation}
$x = x_j$为可去奇点,该处函数值为$w_j = 1/A_j$,即积分权值的倒数。
我们需要绘制的函数
\begin{equation}
    \sqrt{\frac{\rho(x)}{\rho(x_j)}}f_j(x)
    = \frac{\alpha_{n-1}}{\rho(x_j)} \frac{\sqrt{\rho(x_j)}\mathrm{H}_{n-1}(x_j)\sqrt{\rho(x)}\mathrm{H}_n(x)}{x - x_j},
\end{equation}

由于常数系数并不重要,而且这里的$\rho(x_j) = \exp(-x_j^2)$只会导致溢出,因此,仅绘制$\frac{\sqrt{\rho(x_j)}\mathrm{H}_{n-1}(x_j)\sqrt{\rho(x)}\mathrm{H}_n(x)}{x - x_j}$的图像。
此函数仅在$x_j$附近有较大起伏,故仅绘制$x_j\pm 10.0$的图像。
数据输出的代码如下(仅命令行传入参数\texttt{oh}时输出)
{
    \linespread{1.0}
    \lstinputlisting[linerange=end:prob2.3_hermite_weighted-end:prob2.3_hermite_output]{1_Orthogonal_Polynomials.cpp}
}
使用\textsf{Python}绘图(见\texttt{1\_graph.py}),见\autoref{fig:hermite}。
其展现出了$\delta$函数的性态。

\begin{figure}
    \centering
    \subfloat[$j=2$]{
        \label{fig:hermite_2}
        \includegraphics[width=0.48\textwidth]{figures/hermite_weighted_2.pdf}
    }
    \subfloat[$j=16$]{
        \label{fig:hermite_16}
        \includegraphics[width=0.48\textwidth]{figures/hermite_weighted_16.pdf}
    }\\
    \subfloat[$j=128$]{
        \label{fig:hermite_128}
        \includegraphics[width=0.48\textwidth]{figures/hermite_weighted_128.pdf}
    }
    \subfloat[$j=1024$]{
        \label{fig:hermite_1024}
        \includegraphics[width=0.48\textwidth]{figures/hermite_weighted_1024.pdf}
    }
    \caption{厄密多项式$\mathrm{H}_{1024}(x)$第$j$个插值多项式在多项式第$j$个零点附近的图像(标度有放缩以避免溢出)}
    \label{fig:hermite}
\end{figure}

\subsubsection{切比雪夫多项式}
切比雪夫多项式的行为更为良态。
我们可以直接计算出
\begin{equation}
    \sqrt{\frac{\rho(x)}{\rho(x_j)}}f_j(x)
    = \frac{1}{2} \sqrt{\frac{1-x_j^2}{1-x^2}} \frac{\mathrm{T}_{n-1}(x_j)\mathrm{T}_n(x)}{x - x_j}.
\end{equation}

计算高斯积分系数以及绘图数据输出代码如下
{
    \linespread{1.0}
    \lstinputlisting[linerange=end:prob2.3_hermite_output-end:prob2.3]{1_Orthogonal_Polynomials.cpp}
}

高斯积分系数输出为
{
\footnotesize
\begin{verbatim}
--------------------------
j'th Gaussian integration weight of Chebyshev polynomial of degree 1024 T_n(x)
j=      2                       16                      128                     1024
        0.003067961587519458    0.003067961575658197    0.00306796157581011     0.00306796176444558
\end{verbatim}
}
与解析解$\frac{\pi}{1024} \approx 0.003\,067\,961\,57$都特别接近,仅靠近区间边界的稍有偏离。

插值多项式绘图见\autoref{fig:chebyshev},其展现出了$\delta$函数性态。

\begin{figure}
    \centering
    \subfloat[$j=2$]{
        \label{fig:chebyshev_2}
        \includegraphics[width=0.48\textwidth]{figures/chebyshev_weighted_2.pdf}
    }
    \subfloat[$j=16$]{
        \label{fig:chebyshev_16}
        \includegraphics[width=0.48\textwidth]{figures/chebyshev_weighted_16.pdf}
    }\\
    \subfloat[$j=128$]{
        \label{fig:chebyshev_128}
        \includegraphics[width=0.48\textwidth]{figures/chebyshev_weighted_128.pdf}
    }
    \subfloat[$j=1024$]{
        \label{fig:chebyshev_1024}
        \includegraphics[width=0.48\textwidth]{figures/chebyshev_weighted_1024.pdf}
    }
    \caption{切比雪夫多项式$\mathrm{T}_{1024}(x)$第$j$个插值多项式在多项式第$j$个零点附近的图像}
    \label{fig:chebyshev}
\end{figure}

\section{希尔伯特空间}
前面求式 \eqref{eq:mat_eig_simple} 中的矩阵$\vb{T}$的本征值其实就是将其对角化为$\vb{U}\vb{D}\vb{U}^\dagger$,其中$\vb{D}$对角,$\vb{U}$为酉矩阵。
那么又根据$x_j$为$Q_n(x)$的第$j$个零点,我们可以将式 \eqref{eq:mat_eig_simple} 化为
\begin{equation*}
    x_j\vb{U}^\dagger\vb{Q}(x_j) = \vb{D} \vb{U}^\dagger \vb{Q}(x_j),
\end{equation*}
记$g_i(x) = \qty(\vb{U}^\dagger\vb{Q}(x))_i$,那么
\begin{equation*}
    x_jg_i(x_j) = \sum_i x_k\delta_{ki}g_i(x_j) = x_ig_i(x_j),
\end{equation*}
从而$g_i(x_j) = 0,\ i\neq j$,其为离散算符$x_i$的本征函数。
另外,它也满足插值多项式的要求。
因此,根据插值多项式的唯一性,我们可以得到:
\begin{enumerate}
    \item $g_i(x)$与前面定义的$f_i(x)$至多差一个常系数;
    \item $f_i(x)$与$g_i(x)$一样,是离散算符$x_i$的本征函数。
\end{enumerate}
如果我们记$\ket{x_i} = f_i(x)$,那么
\begin{multline}
    \braket{x_i}{x_j} = \braket{\qty(\vb{U}^\dagger\vb{Q}(x))_i}{\qty(\vb{U}^\dagger\vb{Q}(x))_j} \\
    = \sum_k u_{ki}\bra{Q_k} \sum_l u^*_{lj}\ket{Q_l} =  \sum_k u^*_{kj} u_{ki} = (\vb{U}^\dagger\vb{U})_{ji} = \delta_{ij}.
\end{multline}

下面考虑将$x_i$连续化。
为此,我们先定义半连续的
\begin{equation}
    \ket{x}_D = \ket{x_i},\quad x_i\text{ 满足 }\frac{x_i+x_{i-1}}{2} \le x < \frac{x_i+x_{i+1}}{2}.
\end{equation}
那么,在$n\to\infty$,零点变得越发稠密时,
\begin{multline}
    \int_c^d {}_D\braket{x}{y}_D\dd{x} = \sum_{i = k}^{l}\int_\frac{x_i+x_{i-1}}{2}^\frac{x_i+x_{i+1}}{2} \braket{x_i}{x_j} \dd{x}\\
    = \sum_{i = k}^{l} \braket{x_i}{x_j} \Delta x_i = \begin{cases}
        \Delta x_j,&    y\text{ 在积分区间内},\\
        0,&    y\text{ 不在积分区间内}.
    \end{cases}
\end{multline}
其中$k$为最接近$c$的零点编号,$l$为最接近$d$的零点编号,$j$为最接近$y$的零点编号,$\Delta x_i = \frac{x_{i+1}-x_{i-1}}{2}$。

我们再这样定义
\begin{equation}
    \ket{x} = \lim_{n\to\infty}\frac{\ket{x}_D}{\sqrt{\Delta x_i}},
\end{equation}
那么,我们就得到
\begin{equation}
    \int_c^d \braket{x}{y}\dd{x} = \begin{cases}
        1,&    y\text{ 在积分区间内},\\
        0,&    y\text{ 不在积分区间内}.
    \end{cases}
\end{equation}
即$\braket{x}{y} = \delta(x-y)$。

下面我们以勒让德多项式为例来验证一下连续极限。

\subsection{勒让德多项式在\texorpdfstring{$n\to\infty$}{n→∞}的连续极限}
我们期望,$\braket{x_i}{x}$在零点数$n\to\infty$时趋于$\delta(x - x_i)$,即
\begin{equation}
    \int_a^b \rho(\xi)f_i(\xi)f_x{\xi}\dd{x}\to \delta(x-x_i).
\end{equation}
我们可以通过验证
\begin{enumerate}
    \item $n\to\infty$时零点(也就是$\hat{x}$的本征谱)的稠密性,和
    \item $\sqrt{\rho(x)}f_i(x)$在$n\to\infty$时仅在$x\approx x_i$时非零
\end{enumerate}
来验证这一渐进行为。

\subsubsection{零点的稠密性}
我们研究阶数为$16,\ 64,\ 256,\ 1024$的勒让德多项式。
我们期待其零点在定义区间$[-1, 1]$上越发稠密。

为计算它们的零点,我们需要获得三对角矩阵。
归一化的勒让德有
\begin{equation}
    \alpha_n = \frac{n+1}{\sqrt{(2n+1)(2n+3)}},\quad \beta_n = 0.
\end{equation}
所以计算矩阵代码如下
{
    \linespread{1.0}
    \lstinputlisting[linerange=beg:legendre_mat_dec-end:legendre_mat_dec]{1_Orthogonal_Polynomials.cpp}
    \lstinputlisting[linerange=beg:legendre_mat-end:legendre_mat]{1_Orthogonal_Polynomials.cpp}
}

在有命令行参数\texttt{olz}时计算它们的零点并输出的代码如下
{
    \linespread{1.0}
    \lstinputlisting[linerange=end:prob2.3-end:prob3.1_zeroes]{1_Orthogonal_Polynomials.cpp}
}

将零点以编号$j$为横坐标,值为纵坐标绘图于\autoref{fig:legendre_zeroes}。
说明,随着$n$的增加,算符$\hat{x}$的本征值趋于无穷,本征谱连续化。

\begin{figure}
    \centering
    \subfloat[$n=16$]{
        \label{fig:legendre_zeroes_16}
        \includegraphics[width=0.48\textwidth]{figures/legendre_zeroes_16.pdf}
    }
    \subfloat[$n=64$]{
        \label{fig:legendre_zeroes_64}
        \includegraphics[width=0.48\textwidth]{figures/legendre_zeroes_64.pdf}
    }\\
    \subfloat[$n=256$]{
        \label{fig:legendre_zeroes_256}
        \includegraphics[width=0.48\textwidth]{figures/legendre_zeroes_256.pdf}
    }
    \subfloat[$n=1024$]{
        \label{fig:legendre_zeroes_1024}
        \includegraphics[width=0.48\textwidth]{figures/legendre_zeroes_1024.pdf}
    }
    \caption{$n$阶勒让德多项式$\mathrm{P}_{n}(x)$的零点值$x_j$与编号$j$的关系。说明随着$n$的增加,算符$\hat{x}$的本征值连续化}
    \label{fig:legendre_zeroes}
\end{figure}

\subsubsection{本征函数趋于\texorpdfstring{$\delta$}{δ}函数}
勒让德多项式的$\rho(x) = 1$,故我们直接绘制
\begin{equation}
    f_j(x) = \alpha_{n-1} \frac{\mathrm{P}_{n-1}(x_j)\mathrm{P}_n(x)}{x - x_j},
\end{equation}
即可。
对于$n=16,\ 64,\ 256,\ 1024$,分别绘制$f_{6}, f_{28}, f_{110}, f_{400}$的图像。

生成绘图数据的代码见下
{
    \linespread{1.0}
    \lstinputlisting[linerange=end:prob3.1_zeroes-end:prob3.1]{1_Orthogonal_Polynomials.cpp}
}
绘得的图片见\autoref{fig:legendre_poly}。
可以发现,$n=1024$时,$f_{400}(x)$已十分接近一个$\delta$函数。

\begin{figure}
    \centering
    \subfloat[$n=16, j=6$]{
        \label{fig:legendre_poly_16}
        \includegraphics[width=0.48\textwidth]{figures/legendre_poly_16_6.pdf}
    }
    \subfloat[$n=64, j=28$]{
        \label{fig:legendre_poly_64}
        \includegraphics[width=0.48\textwidth]{figures/legendre_poly_64_28.pdf}
    }\\
    \subfloat[$n=256, j=110$]{
        \label{fig:legendre_poly_256}
        \includegraphics[width=0.48\textwidth]{figures/legendre_poly_256_110.pdf}
    }
    \subfloat[$n=1024, j=400$]{
        \label{fig:legendre_poly_1024}
        \includegraphics[width=0.48\textwidth]{figures/legendre_poly_1024_400.pdf}
    }
    \caption{$n$阶勒让德多项式的第$j$个插值多项式。其表现为$\delta$函数$\delta(x-x_j)$}
    \label{fig:legendre_poly}
\end{figure}

\subsection{勒让德多项式的对易子}
由于一般的正交多项式的算符$-\dv{x}$需要知道微分方程,所以其对易子矩阵$[x, -\dv{x}]$并不是很好求。
我们仅考虑归一化的勒让德多项式。

其有微分递推关系
\begin{equation}
    \sqrt{2n-1}\mathrm{P}_{n-1} = \frac{\mathrm{P}'_n}{\sqrt{2n+1}} - \frac{\mathrm{P}'_{n-1}}{\sqrt{2n-3}}.
\end{equation}
很容易就可以得到,
\begin{equation}
    \frac{\mathrm{P}'_n}{\sqrt{2n+1}} = \sqrt{2n-1}\mathrm{P}_{n-1} + \sqrt{2n-5}\mathrm{P}_{n-3} + \cdots.
\end{equation}
从而我们可以写出
\begin{equation}
    \dv{x} \mqty(
        \mathrm{P}_0\\   \mathrm{P}_1\\  \mathrm{P}_2\\    \mathrm{P}_3\\  \vdots
    )
    = \mqty(
        \sqrt{1}\times&\mqty{0&0&0&0&\cdots}\\
        \sqrt{3}\times&\mqty{\sqrt{1}&0&0&0&\cdots}\\
        \sqrt{5}\times&\mqty{0&\sqrt{3}&0&0&\cdots}\\
        \sqrt{7}\times&\mqty{\sqrt{1}&0&\sqrt{5}&0&\cdots}\\
        \vdots&\vdots\\
    ) \mqty(
        \mathrm{P}_0\\   \mathrm{P}_1\\  \mathrm{P}_2\\    \mathrm{P}_3\\  \vdots
    ).
\end{equation}

对任一函数$f(x)$,我们总可以将其展开到仅取到$\mathrm{P}_{n-1}$为止的$n$个正交多项式张成的函数空间,
\begin{equation}
    f(x) = \sum_{k=0}^{n-1} c_k \mathrm{P}_k(x) = \vb{c}^T\vb{P}.
\end{equation}

那么,$x$或$\dv{x}$作用在$f$上时有
\begin{gather}
    xf(x) = \vb{c}^T (x\vb{P}) = \vb{c}^T \vb{T} \vb{P},\\
    \dv{x}f(x) = \vb{c}^T (\dv{x}\vb{P}).
\end{gather}
$\vb{T}$为之前的三对角递推矩阵。
我们可以将$x$或$\dv{x}$理解为作用在系数矢量$\vb{c}$上,则
\begin{gather}
    x\vb{c} = \vb{T}^T\vb{c} = \mqty(
        0&    \frac{1}{\sqrt{3}}&&&&\\
        \frac{1}{\sqrt{3}}& 0&  \frac{2}{\sqrt{15}}&&&\\
        &   \frac{2}{\sqrt{15}}&   0&\frac{3}{\sqrt{35}}&&\\
        &   &   \frac{3}{\sqrt{35}}&    0&\ddots&\\
        &&& \ddots&\ddots&
    )^T\vb{c},\\
    \dv{x}\vb{c} = \mqty(
        0&    &&&&\\
        \sqrt{3}& 0&&&&\\
        0&   \sqrt{15}&   0&&&\\
        \sqrt{7}& 0 &   \sqrt{35}&    0&&\\
        \vdots&\vdots&\vdots& \vdots&\ddots&
    )^T \vb{c}.
\end{gather}

在仅取到$\mathrm{P}_{n-1}$为止的$n$个正交多项式张成的函数空间、仅取相应$\mathrm{P}_{n}$的$n$个零点作为离散的坐标点时,我们可以将作用在系数矢量上的算符$\hat{x}$与$-\dv{\hat{x}}$写成矩阵:
\begin{equation}
    \mqty(
        0&    \frac{1}{\sqrt{3}}&&&&\\
        \frac{1}{\sqrt{3}}& 0&  \frac{2}{\sqrt{15}}&&&\\
        &   \frac{2}{\sqrt{15}}&   0&\frac{3}{\sqrt{35}}&&\\
        &   &   \frac{3}{\sqrt{35}}&    0&\ddots&\\
        &&& \ddots&\ddots&
    )^T,\quad \mqty(
        0&    &&&&\\
        -\sqrt{3}& 0&&&&\\
        0&   -\sqrt{15}&   0&&&\\
        -\sqrt{7}& 0 &   -\sqrt{35}&    0&&\\
        \vdots&\vdots&\vdots& \vdots&\ddots&
        )^T.
\end{equation}

因此,我们要计算
\begin{equation}
    \hat{x}\times -\dv{\hat{x}} =
    -\qty(\mqty(
        0&    &&&&\\
        \sqrt{3}& 0&&&&\\
        0&   \sqrt{5}&   0&&&\\
        \sqrt{7}& 0 &   \sqrt{7}&    0&&\\
        \vdots&\vdots&\vdots& \vdots&\ddots&
    )\mqty(
        0&    \frac{1}{\sqrt{3}}&&&&\\
        \frac{1}{\sqrt{1}}& 0&  \frac{2}{\sqrt{5}}&&&\\
        &   \frac{2}{\sqrt{3}}&   0&\frac{3}{\sqrt{7}}&&\\
        &   &   \frac{3}{\sqrt{5}}&    0&\ddots&\\
        &&& \ddots&\ddots&
    ))^T
\end{equation}
与
\begin{equation}
    -\dv{\hat{x}}\times \hat{x} =
    -\qty(\mqty(
        0&    \frac{1}{\sqrt{1}}&&&&\\
        \frac{1}{\sqrt{3}}& 0&  \frac{2}{\sqrt{3}}&&&\\
        &   \frac{2}{\sqrt{5}}&   0&\frac{3}{\sqrt{5}}&&\\
        &   &   \frac{3}{\sqrt{7}}&    0&\ddots&\\
        &&& \ddots&\ddots&
    )\mqty(
        0&    &&&&\\
        \sqrt{1}& 0&&&&\\
        0&   \sqrt{3}&   0&&&\\
        \sqrt{1}& 0 &   \sqrt{5}&    0&&\\
        \vdots&\vdots&\vdots& \vdots&\ddots&
    ))^T
\end{equation}
之差。

经过矩阵运算可以得到,
\begin{gather}
    \hat{x}\times -\dv{\hat{x}} =
    -\spmqty{
        0 &  &  &  &  &  &  &  \\
        0 & 1 &  &  &  &  &  &  \\
        \sqrt{5\times 1} & 0 & 2 &  &  &  &  &   \\
        0 & \sqrt{7\times 3} & 0 & 3 &  &  &  &   \\
        \sqrt{5\times 1} & 0 & \sqrt{5\times 1} & 0 & 4 &  &  &  \\
        \vdots & \vdots & \vdots & \vdots & \vdots & \ddots & & \\
        &  & & \ddots & {\scriptscriptstyle\sqrt{(2n-3)(2n-7)}} & 0 &  n-2 &  \\
        &  & & \cdots & 0 & {\scriptscriptstyle\sqrt{(2n-1)(2n-5)}} & 0 &  n-1
    }^T,\\
    -\dv{\hat{x}}\times \hat{x} =
    - \spmqty{
        1 &  &  &  &  &  &  &  \\
        0 & 2 &  &  &  &  &  &  \\
        \sqrt{5\times 1} & 0 & 3 &  &  &  &  &  \\
        0 & \sqrt{7\times 3} & 0 & 4 &  &  &  &  \\
        \sqrt{9\times 1} & 0 & \sqrt{9\times 5} & 0 & 5 &  &  &  \\
        \vdots & \vdots & \vdots & \vdots & \ddots & \ddots &  &  \\
        &  &  & \ddots & {\scriptscriptstyle\sqrt{(2n-3)(2n-7)}} & 0 &  n-1 &  \\
        &  &  & \cdots & 0 & {\scriptscriptstyle(n-1)\sqrt{\frac{2n-5}{2n-1}}} & 0 & 0
    }^T.
\end{gather}

两者相减,我们可以得到一个几乎与单位矩阵$\vb{I}$相同的矩阵——除了其中一列。
我们有
\begin{equation}
    \qty[\hat{x}, -\dv{\hat{x}}]
    = \hat{x}\times -\dv{\hat{x}} - \qty(-\dv{\hat{x}}\times \hat{x})
    = \vb{I} - n \mqty(
        \cdots &
        \mqty{
            \vdots\\
            0\\
            \sqrt{\frac{2n-9}{2n-1}}\\
            0\\
            \sqrt{\frac{2n-5}{2n-1}}\\
            0\\
            1
        }
    ).
\end{equation}

对任一Krylov子空间维度,对易子与单位算符的差别仅在于一列上。
在维数趋于无穷时,两者趋于相等。


% \appendix
% \renewcommand\chapterendname{}

% \chapter{\texttt{Misc}数值库接口}
% 本人将一些基本的矩阵类型以及相关算法写为\textsf{C++}头文件以方便调用。
本附录将简要介绍可用的接口。
所有的类以及函数均定义于命名空间\texttt{Misc}中,故下文不将\texttt{Misc::}显式写出。

可以通过引入\texttt{"Matrix\_Catalogue.h"}头文件引入所有矩阵类型以及有关运算符重载。
可以通过引入\texttt{"linear\_eq\_direct.h"}与\texttt{"linear\_eq\_iterative.h"}头文件调用矩阵的直接解法与迭代解法。

\section{矩阵}
所有的矩阵类型都继承自虚基类\texttt{template <typename T, size\_t R, size\_t C> class Base\_Matrix},其中模板参数\texttt{T}为数据类型,\texttt{R}, \texttt{C}分别为行数和列数。
注意,\textbf{所有矩阵规模参数都需要编译时指定}。
继承树为
{
\small
\begin{verbatim}
Base_Matrix<T, R, C> 虚基类
|Band_Matrix<T, N, M> 长宽为N,半带宽为M的带状矩阵
||Base_Half_Band_Matrix<T, N, M> 虚基类,长宽为N,半带宽为M
|||Low_Band_Matrix<T, N, M> 长宽为N,半带宽为M的下三角带状矩阵
|||Symm_Band_Matrix<T, N, M> 长宽为N,半带宽为M的对称带状矩阵
|||Up_Band_Matrix<T, N, M> 长宽为N,半带宽为M的上三角带状矩阵
|Base_Tri_Matrix<T, N> 虚基类,长宽为N
||Hermite_Matrix<T, N> 厄密矩阵,长宽为N
||Low_Tri_Matrix<T, N> 下三角矩阵,长宽为N
||Symm_Matrix<T, N> 对称矩阵,长宽为N
||Up_Tri_Matrix<T, N> 上三角矩阵,长宽为N
|Diag_Matrix<T, N> 对角矩阵,长宽为N
|Matrix<T, R, C> 一般矩阵,R行C列
|Sparse_Matrix<T, R, C> 稀疏矩阵;同时继承自std::array<std::map<size_t, T>, R>
\end{verbatim}
}

所有矩阵均提供的接口有
{
\linespread{1.0}
\begin{lstlisting}
Row<T, R, C> row(size_type pos);
const Row<T, R, C> row(size_type pos) const; // 返回某一行
Row<T, R, C> operator[](size_type pos);
const Row<T, R, C> operator[](size_type pos) const; // 同row()
Column<T, R, C> column(size_type pos);
const Column<T, R, C> column(size_type pos) const; // 返回某一列

const Transpose<T, C, R> trans() const; // 返回仅作为右值的转置

T &operator()(size_type row, size_type col);
const T &operator()(size_type row, size_type col); // 返回元素

using p_ta = T (*)[C];
p_ta data();
const p_ta data() const; // 返回内部存储数据的指针
constexpr size_type size() const; // 返回作为一个矩阵的元素个数
size_type data_size() const; // 返回存储的元素个数
size_type data_lines() const; // 返回存储的元素行数
constexpr size_type rows() const; // 返回矩阵行数
constexpr size_type cols() const; // 返回矩阵列数
std::pair<size_type, size_type> shape() const; // 返回矩阵形状
\end{lstlisting}
}

\texttt{row()}和\texttt{column()}方法返回\texttt{Row<T, R, C>}或\texttt{Column<T, R, C>}对象,可作为左值或右值。
可以通过\texttt{[]}下标运算符取或修改原矩阵的元素。

使用标准库类型\texttt{std::array<T, N>}作为矢量。

重载了矩阵间、矩阵与\texttt{Row}或\texttt{Column}或\texttt{array}之间的\texttt{*}运算符,可进行矩阵乘法。

所有矩阵均支持默认初始化,可以提供一个默认值。
如不提供,默认为0。

定义了一些类型间转换,可以把一些特殊的矩阵转化为更一般的矩阵。
比如,\texttt{Diag\_Matrix}可以转换为\texttt{Band\_Matrix}、\texttt{Up\_Tri\_Matrix}等等矩阵。

同时还定义了通过初始化器的列表初始化以便字面指定矩阵。
下面主要介绍一下这一初始化的使用。

\subsection{带状矩阵\texttt{Band\_Matrix}等}
需要沿对角线方向从右上到左下逐行输入,比如
{
\small
\begin{verbatim}
{
    {4, 6, 8, 3, 7},
    {3, 4, 7, 7, 4, 4},
    {2, 5, 5, 5, 2, 1, 7},
    {6, 3, 4, 4, 3, 6},
    {2, 7, 9, 2, 5}
}
\end{verbatim}
}
会得到矩阵\[
    \mqty(
        2&3&4&0&0&0&0\\
        6&5&4&6&0&0&0\\
        2&3&5&7&8&0&0\\
        0&7&4&5&7&3&0\\
        0&0&9&4&2&4&7\\
        0&0&0&2&3&1&4\\
        0&0&0&0&5&6&7
    ).
\]
输入格式错误会引发运行时错误。

对于半带状矩阵以及对称带状矩阵,初始化器的输入顺序为\textbf{从对角线向非零一侧}。

\subsection{三角矩阵\texttt{Base\_Tri\_Matrix}等}
水平从上到下依次输入半侧元素。
比如,
{
\small
\begin{verbatim}
{
    {4},
    {3, 4},
    {2, 1, 7},
    {6, 3, 4, 6},
    {2, 7, 9, 2, 5}
}
\end{verbatim}
}
会得到下三角矩阵\[
    \mqty(
        4&&&&\\
        3&4&&&\\
        2&1&7&&\\
        6&3&4&6&\\
        2&7&9&2&5
    )
\]
或对称矩阵\[
    \mqty(
        4&3&2&6&2\\
        3&4&1&3&7\\
        2&1&7&4&9\\
        6&3&4&6&2\\
        2&7&9&2&5
    ).
\]

但对应的上三角矩阵需要这样输入,
{\small
\begin{verbatim}
{
    {2, 7, 9, 2, 5},
    {6, 3, 4, 6},
    {2, 1, 7},
    {3, 4},
    {4}
}
\end{verbatim}
}
得到\[
    \mqty(
        2&7&9&2&5\\
        0&6&3&4&6\\
        0&0&2&1&7\\
        0&0&0&3&4\\
        0&0&0&0&4
    ).
\]

\subsection{对角矩阵\texttt{Diag\_Matrix}}
直接在一个初始化器列表内输入所有对角元即可。
也可通过提供一对迭代器初始化。

\section{矩阵的直接解法}
\texttt{"linear\_eq\_direct.h"}中主要是对矩阵进行分解的算法,提供\begin{enumerate}
    \item 三角矩阵回代 \texttt{back_sub()}
    \item LU分解 \texttt{lu\_factor()}
    \item LDL分解 \texttt{ldl\_factor()}
    \item Gauss-Jordan法求逆矩阵 \texttt{inv()};对特殊矩阵也会调用用其他算法
    \item 三对角矩阵追赶法 \texttt{tri\_factor()}
    \item Cholesky分解 \texttt{cholesky}
    \item 行列式计算 \texttt{det()}
\end{enumerate}

所有算法(除\texttt{det()})外均提供原处修改的接口与非原处修改接口。
如果仅传入待分解矩阵,那么会进行原处修改;如果也传入了输出矩阵,那么不会修改原矩阵。

\texttt{det()}函数会返回行列式的值。

\section{矩阵的迭代解法}
\texttt{"linear\_eq\_iterative.h"}中主要是对线性方程组进行迭代求解的算法,提供\begin{enumerate}
    \item Jacobi迭代法 \texttt{jacobi()}
    \item Gauss-Seidel迭代法 \texttt{gauss\_seidel()}
    \item 超松弛迭代法 \texttt{suc\_over\_rel()}
    \item 共轭梯度法 \texttt{grad\_des()}
\end{enumerate}

所有算法的接口是基本统一的,参数列表如下\begin{enumerate}
    \item \verb|const Base_Matrix<T, N, N> &in_mat| 系数矩阵
    \item \verb|const std::array<T, N> &in_b| 非齐次项
    \item \verb|std::array<T, N> &out_x| 解向量的初始值;输出的解向量
    \item \verb|T omega| 仅超松弛迭代使用的系数
    \item \verb|bool sparse = false| 是否为稀疏矩阵;如是,那么有优化
    \item \verb|size_t max_times = 1000| 最大迭代次数
    \item \verb|double rel_epsilon = 1e-15| 判停标准:残差与初始残差的1-范数之比小于此值时即停止
\end{enumerate}


\end{document}
